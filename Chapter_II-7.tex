\chapter{قراءة و كتابة الملفات}
المشكل مع استعمال المتغيّرات، هو أنها موجودة فقط في الذاكرة العشوائية
\textenglish{RAM}.
بخروجنا من البرنامج، كلّ المتغيّرات يتم حذفها من الذاكرة و لن يصبح ممكنا إستعادة قيمها. كيف يمكننا إذن أن نحتفظ بأحسن العلامات التي تحصّلنا عليها في لعبة ؟ كيف يمكننا إنشاء محرر نصوص إذا كان كلّ النصّ  المكتوب يختفي بمجرّد إيقاف البرنامج ؟

لحسن الحظّ يمكننا القراءة من الملفاّت و كذا الكتابة فيها في لغة
\textenglish{C}.
هذه الملفّات مُخزّنة في القرص الصلب
(\textenglish{Hard disk})
الخاص بالحاسوب : الشيء الإيجابيّ إذن هو أنها تبقى محفوظة، حتّى عند إيقاف البرنامج أو الحاسوب.

للقراءة من الملفات و الكتابة فيها، سنحتاج إلى استعمال كلّ ما درسناه حتّى الآن : المؤشرات، الهياكل، السلاسل المحرفيّة، الخ.

\section{فتح و غلق ملف}
للقراءة و الكتابة في الملفّات، سنستعمل دوالاً معرّفة في المكتبة
\InlineCode{stdio}
التي استعملناها سابقاً.\\
نعم، هذه المكتبة تحتوي على الدالتين
\InlineCode{scanf}
و
\InlineCode{printf}
اللتان نعرفهما جيّدا ! لكن ليس هذا فحسب : يوجد بها الكثير من الدوال الأخرى، خصوصا التي تعمل على الملفات.

\begin{information}
  كل المكتبات التي استعملناها حتّى الآن
(\InlineCode{stdlib.h}، \InlineCode{stdio.h}، \InlineCode{math.h}، \InlineCode{string.h}...)
تشكّل ما نسميه بالمكتبات القياسية
(\textenglish{standard libraries})،
و هي مكتبات تأتي تلقائيا مع البيئة التطويرية التي تستخدمها و لديها الميزة في أنّها تعمل على كل أنظمة التشغيل. بالإمكان استعمالها في أيّ مكان، سواء كنت في
\textenglish{Windows}،
أو
\textenglish{GNU/Linux}
أو
\textenglish{Mac}
أو غير ذلك.
المكتبات القياسيّة ليست كثيرة و لا تمكّننا من القيام بأكثر من بعض الأمور الأساسيّة، كما فعلنا لغاية الآن. للحصول على وظائف أكثر تقدّما، كفتح النوافذ، يجب تحميل و تثبيت مكتبات جديدة. سنرى ذلك قريبا !
\end{information}

تأكّد إذن، للبدأ، أن تقوم بتضمين المكتبتين
\InlineCode{stdio.h}
و
\InlineCode{stdlib.h}
على الأقل أعلى ملفكم
\InlineCode{.c} :

\begin{Csource}
#include <stdlib.h>
#include <stdio.h>
\end{Csource}

هاتان المكتبتان ضروريتان و أساسيّتان لدرجة أنّي أنصحك بتضمينهما في كلّ البرامج التي تكتبها في المستقبل، أيّا كانت.

حسناً و بعدما قمنا بتضمين المكتبتين، يمكننا أن ننطلق في بالأمور الجدّيّة. إليك الخطوات التي يجب إتّباعها دائماً حينما تريد العمل على ملف، سواء للقراءة منه أو للكتابة فيه :
\begin{itemize}
  \item نقوم بمناداة دالة
\textbf{فتح الملف}
\InlineCode{fopen}
التي تقوم بإرجاع مؤشّر نحو هذا الملف.
  \item \textbf{نتأكّد من نجاح عمليّة الفتح}
(أي إن كان الملفّ موجودا) باختبار قيمة المؤشر الذي أرجعته الدالة. فإن كان المؤشر يساوي
\InlineCode{NULL}،
فهذا يعني أنّ فتح الملف لم ينجح، في هذه الحالة لا يمكننا الإكمال (يجب أن نظهر رسالة خطا).
  \item إذا تم الفتح بنجاح (أي أن قيمة المؤشر تختلف عن
\InlineCode{NULL})،
سنستمتع
\textbf{بالكتابة على الملف أو القراءة منه}،
و ذلك باستخدام دوال سنراها لاحقاً.
  \item بمجرّد أن
\textbf{ننهي العمل على الملف}،
يجب تذكّر "غلقه" باستعمال الدالة
\InlineCode{fclose}.
\end{itemize}
سنتعلّم كخطوة أولى كيف نستخدم
\InlineCode{fopen}
و
\InlineCode{fclose}،
حينما تتعلّم هذا، سنتعلّم كيف نقرأ محتواه و نكتب نصّا فيه.

\subsection{\texttt{fopen} : فتح ملف}
في فصل السلاسل المحرفيّة، كنا نستعين بنماذج الدوال مثل "دليل استخدام". هذا ما يفعله المبرمجون غالبا : يقرؤون نموذج دالة و يفهمون كيف يستخدمونها. مع ذلك، أعلم أنّنا بحاجة إلى بعض الشروحات البسيطة !

لهذا فلنرى قليلاً نموذج
\InlineCode{fopen} :

\begin{Csource}
FILE* fopen(const char* fileName, const char* openMode);
\end{Csource}

هذه الدالة تنتظر معاملين :
\begin{itemize}
  \item اسم الملف الذي نريد فتحه.
  \item أسلوب فتح الملف، أي دلالة تذكر ما الّذي تريد فعله : القراءة من الملف، أو الكتابة فيه، أو كليهما.
\end{itemize}

هذه الدالة ترجع... مؤشّرا على
\InlineCode{FILE} !
إنّه مؤشّر على هيكل من نوع
\InlineCode{FILE}.
هذا الهيكل متواجد في المكتبة
\InlineCode{stdio.h}.
يمكنك فتح الملف لترى مما يتكوّن النوع
\InlineCode{FILE}،
لكن هذا ليس ما يهمّنا.

\begin{question}
  لكن لِمَ اسم الهيكل كله بأحرف كبيرة؟ اعتقدت أن الأسماء بالأحرف الكبيرة حجزناها للثوابت و لـ\InlineCode{\#define} ؟
\end{question}

هذه "القاعدة"، أنا من قمت بتحديدها (و كثير من المبرمجين يتيعونها)، و لكنّها لم تكن أبدا مفروضة. و يبدو أنّ من برمجوا
\InlineCode{stdio.h}
لا يتبعون نفس القواعد !\\
هذا لا يجب أن يشوّشك كثيرا. سوف ترى أنّ المكتبات الّتي سندرسها لاحقا تتبّع نفس القواعد التي أتّبعها، أي أن اسم الهيكل يبتدئ فقط بحرف واحد كبير.

لنعد إلى دالتنا
\InlineCode{fopen}،
إنها تقوم بارجاع
\InlineCode{FILE*}.
إنه من المهم جدّا استرجاع هذا المؤشّر كي نتمكّن لاحقاً من القراءة و الكتابة في الملف.  و لهذا سنقوم بإنشاء مؤشّر على
\InlineCode{FILE}،
في بداية دالتنا
(\InlineCode{main}
مثلا) :

\begin{Csource}
int main(int argc, char *argv[])
{
	FILE* file = NULL;
	return 0;
}
\end{Csource}

لقد هيّأنا المؤشّر على
\InlineCode{NULL}
من البداية. أذكّرك بأنّ هذه قاعدة أساسيّة أن تهيّأ كلّ المؤشّرات على
\InlineCode{NULL}
إنّ لم تكن لديك قيمة أخرى لإعطائها. إن لم تفعل ذلك، فأنت تزيد كثيرا خطر وجود أخطاء لاحقا.

\begin{information}
  إنه ليس ضرورياً أن تكتب
\InlineCode{struct FILE* file = NULL}،
لأن منشئي
\InlineCode{stdio.h}
قد وضعوا
\InlineCode{typedef}
كما علّمتك منذ مدّة قصيرة.
لاحظ أن شكل الهيكل قد يتغيّر من نظام تشغيل إلى آخر (لا تملك بالضرورة نفس المركّبات في كل الأنظمة). لهذا فلن نعدّل محتوى
\InlineCode{FILE}
مباشرة (لا نقوم بـ\InlineCode{file.element}
مثلا). بل سنكتفي باستدعاء دوال، تتعامل مع
\InlineCode{FILE}
نيابة عناً.
\end{information}

الآن سنقوم بمناداة الدالة
\InlineCode{fopen}،
و استرجاع القيمة الّتي تعيدها في المؤشر
\InlineCode{file}.
و لكن قبل هذا يجب أن اشرح لك كيف تستخدم المعامل الثاني
\InlineCode{openMode}.
في الواقع، هناك شفرة تدلّ للحاسوب على أنك تريد أن تفتح الملف بوضع القراءة فقط، الكتابة فقط أو الاثنين معاً.\\
هذه هي أوضاع فتح الملف المختلفة :
\begin{itemize}
  \item \textbf{\textenglish{"r"} :
قراءة فقط
(\textenglish{read only})}.
يمكنك قراءة محتوى الملف، و لكن لا يمكنك الكتابة فيه.
\textit{يجب أن يكون الملف موجوداً من قبل}.
  \item \textbf{\textenglish{"w"} :
كتابة فقط
(\textenglish{write only})}.
يمكنك الكتابة في الملف، لكن لا يمكنك قراءة محتواه.
\textit{إذا لم يكن الملف موجوداً من قبل، فإنه سيتم إنشاؤه}.
  \item \textbf{\textenglish{"a"} :
إلحاق
(\textenglish{append})}.
يمكنك الكتابة في الملف، إنطلاقا من نهايته.
\textit{إن لم يكن الملف موجوداً، فسيتم إنشاؤه}.
  \item \textbf{\textenglish{"r+"} :
قراءة و كتابة
(\textenglish{read and write})}.
يمكنك القراءة من الملف و الكتابة فيه.
\textit{يجب أن يكون الملف موجوداً من قبل}.
  \item \textbf{\textenglish{"w+"} :
قراءة و كتابة مع مسح المحتوى أوّلا}.
سيتم تفريغ الملف من محتواه أولاً، ثم بإمكانك الكتابة فيه و قراءة محتواه بعد ذلك.
\textit{إن لم يكن الملف موجوداً من قبل، سيتم إنشاؤه}.
  \item \textbf{\textenglish{"a+"}
إلحاق مع القراءة / الكتابة في آخر الملف}.
يمكنك القراءة و الكتابة إنطلاقا من نهاية الملف.
\textit{إن لم يكن موجوداً، سيتم إنشاؤه}.
\end{itemize}

لمعلوماتك، أنا عرضت لك بعضا من أوضاع فتح ملف. في الحقيقة، يوجد ضعفها !
من أجل كل وضع رأيناه هنا، إن أضفت
\InlineCode{"b"}
بعد المحرف الأول
(\InlineCode{"rb"}، \InlineCode{"wb"}، \InlineCode{"ab"}، \InlineCode{"rb+"}، \InlineCode{"wb+"}، \InlineCode{"ab+"})،
فإن الملف سيتم فتحه بالوضع الثنائي
(\textenglish{binary}).
هذا وضع خاص قليلاً فلن ندرسه هنا. في الواقع وضع النص يختصّ بتخزين... النص، تماما كما يوحي الاسم (فقط المحارف القابلة للعرض). أما الوضع الثنائي، يسمح بتخزين المعلومات
بايتا بايتا
(\textenglish{Byte by byte})
(أرقام بشكل أساسي). هذا مختلف كثيرا. على أي حال فطريقة العمل هي تقريبا نفس الّتي سنراها هنا.

شخصياً، أستعمل كثيراً الأوضاع :
\InlineCode{"r"}
(قراءة)،
\InlineCode{"w"}
(كتابة)،
\InlineCode{"r+"}
(قراءة و كتابة في آن واحد). وضع
\InlineCode{"w+"}
خطر قليلاً لأنه يقوم بمسح محتوى الملف مباشرة، بدون أن يطلب التأكيد قبل القيام بذلك. إن هذا الوضع ليس مفيداً إلا إذا أردنا أن نعيد تهيئة الملف أوّلا.
وضع الإلحاق
(\InlineCode{"a"})
يمكنه أن يفيد في بعض الحالات، إذا كنت تريد إضافة معلومات إلى نهاية الملف.

\begin{information}
  إن كنت تريد قراءة ملفّ، فمن المستحسن وضع
\InlineCode{"r"}.
بالطبع، الوضع
\InlineCode{"r+"}
يعمل أيضا، لكن بوضع
\InlineCode{"r"}
فأنت تضمن أنّ الملفّ لا يمكن تعديله، هذا نوع من الحماية.
\end{information}
