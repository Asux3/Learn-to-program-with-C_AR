\chapter{قراءة و كتابة الملفات}
المشكل مع استعمال المتغيّرات، هو أنها موجودة فقط في الذاكرة العشوائية
\textenglish{RAM}.
بخروجنا من البرنامج، كلّ المتغيّرات يتم حذفها من الذاكرة و لن يصبح ممكنا إستعادة قيمها. كيف يمكننا إذن أن نحتفظ بأحسن العلامات التي تحصّلنا عليها في لعبة ؟ كيف يمكننا إنشاء محرر نصوص إذا كان كلّ النصّ  المكتوب يختفي بمجرّد إيقاف البرنامج ؟

لحسن الحظّ يمكننا القراءة من الملفاّت و كذا الكتابة فيها في لغة
\textenglish{C}.
هذه الملفّات مُخزّنة في القرص الصلب
(\textenglish{Hard disk})
الخاص بالحاسوب : الشيء الإيجابيّ إذن هو أنها تبقى محفوظة، حتّى عند إيقاف البرنامج أو الحاسوب.

للقراءة من الملفات و الكتابة فيها، سنحتاج إلى استعمال كلّ ما درسناه حتّى الآن : المؤشرات، الهياكل، السلاسل المحرفيّة، الخ.

\section{فتح و غلق ملف}
للقراءة و الكتابة في الملفّات، سنستعمل دوالاً معرّفة في المكتبة
\InlineCode{stdio}
التي استعملناها سابقاً.\\
نعم، هذه المكتبة تحتوي على الدالتين
\InlineCode{scanf}
و
\InlineCode{printf}
اللتان نعرفهما جيّدا ! لكن ليس هذا فحسب : يوجد بها الكثير من الدوال الأخرى، خصوصا التي تعمل على الملفات.

\begin{information}
  كل المكتبات التي استعملناها حتّى الآن
(\InlineCode{stdlib.h}، \InlineCode{stdio.h}، \InlineCode{math.h}، \InlineCode{string.h}...)
تشكّل ما نسميه بالمكتبات القياسية
(\textenglish{standard libraries})،
و هي مكتبات تأتي تلقائيا مع البيئة التطويرية التي تستخدمها و لديها الميزة في أنّها تعمل على كل أنظمة التشغيل. بالإمكان استعمالها في أيّ مكان، سواء كنت في
\textenglish{Windows}،
أو
\textenglish{GNU/Linux}
أو
\textenglish{Mac}
أو غير ذلك.
المكتبات القياسيّة ليست كثيرة و لا تمكّننا من القيام بأكثر من بعض الأمور الأساسيّة، كما فعلنا لغاية الآن. للحصول على وظائف أكثر تقدّما، كفتح النوافذ، يجب تحميل و تثبيت مكتبات جديدة. سنرى ذلك قريبا !
\end{information}

تأكّد إذن، للبدأ، أن تقوم بتضمين المكتبتين
\InlineCode{stdio.h}
و
\InlineCode{stdlib.h}
على الأقل أعلى ملفكم
\InlineCode{.c} :

\begin{Csource}
#include <stdlib.h>
#include <stdio.h>
\end{Csource}

هاتان المكتبتان ضروريتان و أساسيّتان لدرجة أنّي أنصحك بتضمينهما في كلّ البرامج التي تكتبها في المستقبل، أيّا كانت.

حسناً و بعدما قمنا بتضمين المكتبتين، يمكننا أن ننطلق في بالأمور الجدّيّة. إليك الخطوات التي يجب إتّباعها دائماً حينما تريد العمل على ملف، سواء للقراءة منه أو للكتابة فيه :
\begin{itemize}
  \item نقوم بمناداة دالة
\textbf{فتح الملف}
\InlineCode{fopen}
التي تقوم بإرجاع مؤشّر نحو هذا الملف.
  \item \textbf{نتأكّد من نجاح عمليّة الفتح}
(أي إن كان الملفّ موجودا) باختبار قيمة المؤشر الذي أرجعته الدالة. فإن كان المؤشر يساوي
\InlineCode{NULL}،
فهذا يعني أنّ فتح الملف لم ينجح، في هذه الحالة لا يمكننا الإكمال (يجب أن نظهر رسالة خطا).
  \item إذا تم الفتح بنجاح (أي أن قيمة المؤشر تختلف عن
\InlineCode{NULL})،
سنستمتع
\textbf{بالكتابة على الملف أو القراءة منه}،
و ذلك باستخدام دوال سنراها لاحقاً.
  \item بمجرّد أن
\textbf{ننهي العمل على الملف}،
يجب تذكّر "غلقه" باستعمال الدالة
\InlineCode{fclose}.
\end{itemize}
سنتعلّم كخطوة أولى كيف نستخدم
\InlineCode{fopen}
و
\InlineCode{fclose}،
حينما تتعلّم هذا، سنتعلّم كيف نقرأ محتواه و نكتب نصّا فيه.
