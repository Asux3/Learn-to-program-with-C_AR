\chapter{قراءة و كتابة الملفات}
المشكل مع استعمال المتغيّرات، هو أنها موجودة فقط في الذاكرة العشوائية
\textenglish{RAM}.
بخروجنا من البرنامج، كلّ المتغيّرات يتم حذفها من الذاكرة و لن يصبح ممكنا إستعادة قيمها. كيف يمكننا إذن أن نحتفظ بأحسن العلامات التي تحصّلنا عليها في لعبة ؟ كيف يمكننا إنشاء محرر نصوص إذا كان كلّ النصّ  المكتوب يختفي بمجرّد إيقاف البرنامج ؟

لحسن الحظّ يمكننا القراءة من الملفاّت و كذا الكتابة فيها في لغة
\textenglish{C}.
هذه الملفّات مُخزّنة في القرص الصلب
(\textenglish{Hard disk})
الخاص بالحاسوب : الشيء الإيجابيّ إذن هو أنها تبقى محفوظة، حتّى عند إيقاف البرنامج أو الحاسوب.

للقراءة من الملفات و الكتابة فيها، سنحتاج إلى استعمال كلّ ما درسناه حتّى الآن : المؤشرات، الهياكل، السلاسل المحرفيّة، الخ.

\section{فتح و غلق ملف}
للقراءة و الكتابة في الملفّات، سنستعمل دوالاً معرّفة في المكتبة
\InlineCode{stdio}
التي استعملناها سابقاً.\\
نعم، هذه المكتبة تحتوي على الدالتين
\InlineCode{scanf}
و
\InlineCode{printf}
اللتان نعرفهما جيّدا ! لكن ليس هذا فحسب : يوجد بها الكثير من الدوال الأخرى، خصوصا التي تعمل على الملفات.

\begin{information}
  كل المكتبات التي استعملناها حتّى الآن
(\InlineCode{stdlib.h}، \InlineCode{stdio.h}، \InlineCode{math.h}، \InlineCode{string.h}...)
تشكّل ما نسميه بالمكتبات القياسية
(\textenglish{standard libraries})،
و هي مكتبات تأتي تلقائيا مع البيئة التطويرية التي تستخدمها و لديها الميزة في أنّها تعمل على كل أنظمة التشغيل. بالإمكان استعمالها في أيّ مكان، سواء كنت في
\textenglish{Windows}،
أو
\textenglish{GNU/Linux}
أو
\textenglish{Mac}
أو غير ذلك.
المكتبات القياسيّة ليست كثيرة و لا تمكّننا من القيام بأكثر من بعض الأمور الأساسيّة، كما فعلنا لغاية الآن. للحصول على وظائف أكثر تقدّما، كفتح النوافذ، يجب تحميل و تثبيت مكتبات جديدة. سنرى ذلك قريبا !
\end{information}

تأكّد إذن، للبدأ، أن تقوم بتضمين المكتبتين
\InlineCode{stdio.h}
و
\InlineCode{stdlib.h}
على الأقل أعلى ملفكم
\InlineCode{.c} :

\begin{Csource}
#include <stdlib.h>
#include <stdio.h>
\end{Csource}

هاتان المكتبتان ضروريتان و أساسيّتان لدرجة أنّي أنصحك بتضمينهما في كلّ البرامج التي تكتبها في المستقبل، أيّا كانت.

حسناً و بعدما قمنا بتضمين المكتبتين، يمكننا أن ننطلق في بالأمور الجدّيّة. إليك الخطوات التي يجب إتّباعها دائماً حينما تريد العمل على ملف، سواء للقراءة منه أو للكتابة فيه :
\begin{itemize}
  \item نقوم بمناداة دالة
\textbf{فتح الملف}
\InlineCode{fopen}
التي تقوم بإرجاع مؤشّر نحو هذا الملف.
  \item \textbf{نتأكّد من نجاح عمليّة الفتح}
(أي إن كان الملفّ موجودا) باختبار قيمة المؤشر الذي أرجعته الدالة. فإن كان المؤشر يساوي
\InlineCode{NULL}،
فهذا يعني أنّ فتح الملف لم ينجح، في هذه الحالة لا يمكننا الإكمال (يجب أن نظهر رسالة خطا).
  \item إذا تم الفتح بنجاح (أي أن قيمة المؤشر تختلف عن
\InlineCode{NULL})،
سنستمتع
\textbf{بالكتابة على الملف أو القراءة منه}،
و ذلك باستخدام دوال سنراها لاحقاً.
  \item بمجرّد أن
\textbf{ننهي العمل على الملف}،
يجب تذكّر "غلقه" باستعمال الدالة
\InlineCode{fclose}.
\end{itemize}
سنتعلّم كخطوة أولى كيف نستخدم
\InlineCode{fopen}
و
\InlineCode{fclose}،
حينما تتعلّم هذا، سنتعلّم كيف نقرأ محتواه و نكتب نصّا فيه.

\subsection{\texttt{fopen} : فتح ملف}
في فصل السلاسل المحرفيّة، كنا نستعين بنماذج الدوال مثل "دليل استخدام". هذا ما يفعله المبرمجون غالبا : يقرؤون نموذج دالة و يفهمون كيف يستخدمونها. مع ذلك، أعلم أنّنا بحاجة إلى بعض الشروحات البسيطة !

لهذا فلنرى قليلاً نموذج
\InlineCode{fopen} :

\begin{Csource}
FILE* fopen(const char* fileName, const char* openMode);
\end{Csource}

هذه الدالة تنتظر معاملين :
\begin{itemize}
  \item اسم الملف الذي نريد فتحه.
  \item أسلوب فتح الملف، أي دلالة تذكر ما الّذي تريد فعله : القراءة من الملف، أو الكتابة فيه، أو كليهما.
\end{itemize}

هذه الدالة ترجع... مؤشّرا على
\InlineCode{FILE} !
إنّه مؤشّر على هيكل من نوع
\InlineCode{FILE}.
هذا الهيكل متواجد في المكتبة
\InlineCode{stdio.h}.
يمكنك فتح الملف لترى مما يتكوّن النوع
\InlineCode{FILE}،
لكن هذا ليس ما يهمّنا.

\begin{question}
  لكن لِمَ اسم الهيكل كله بأحرف كبيرة؟ اعتقدت أن الأسماء بالأحرف الكبيرة حجزناها للثوابت و لـ\InlineCode{\#define} ؟
\end{question}

هذه "القاعدة"، أنا من قمت بتحديدها (و كثير من المبرمجين يتيعونها)، و لكنّها لم تكن أبدا مفروضة. و يبدو أنّ من برمجوا
\InlineCode{stdio.h}
لا يتبعون نفس القواعد !\\
هذا لا يجب أن يشوّشك كثيرا. سوف ترى أنّ المكتبات الّتي سندرسها لاحقا تتبّع نفس القواعد التي أتّبعها، أي أن اسم الهيكل يبتدئ فقط بحرف واحد كبير.

لنعد إلى دالتنا
\InlineCode{fopen}،
إنها تقوم بارجاع
\InlineCode{FILE*}.
إنه من المهم جدّا استرجاع هذا المؤشّر كي نتمكّن لاحقاً من القراءة و الكتابة في الملف.  و لهذا سنقوم بإنشاء مؤشّر على
\InlineCode{FILE}،
في بداية دالتنا
(\InlineCode{main}
مثلا) :

\begin{Csource}
int main(int argc, char *argv[])
{
	FILE* file = NULL;
	return 0;
}
\end{Csource}

لقد هيّأنا المؤشّر على
\InlineCode{NULL}
من البداية. أذكّرك بأنّ هذه قاعدة أساسيّة أن تهيّأ كلّ المؤشّرات على
\InlineCode{NULL}
إنّ لم تكن لديك قيمة أخرى لإعطائها. إن لم تفعل ذلك، فأنت تزيد كثيرا خطر وجود أخطاء لاحقا.

\begin{information}
  إنه ليس ضرورياً أن تكتب
\InlineCode{struct FILE* file = NULL}،
لأن منشئي
\InlineCode{stdio.h}
قد وضعوا
\InlineCode{typedef}
كما علّمتك منذ مدّة قصيرة.
لاحظ أن شكل الهيكل قد يتغيّر من نظام تشغيل إلى آخر (لا تملك بالضرورة نفس المركّبات في كل الأنظمة). لهذا فلن نعدّل محتوى
\InlineCode{FILE}
مباشرة (لا نقوم بـ\InlineCode{file.element}
مثلا). بل سنكتفي باستدعاء دوال، تتعامل مع
\InlineCode{FILE}
نيابة عناً.
\end{information}

الآن سنقوم بمناداة الدالة
\InlineCode{fopen}،
و استرجاع القيمة الّتي تعيدها في المؤشر
\InlineCode{file}.
و لكن قبل هذا يجب أن اشرح لك كيف تستخدم المعامل الثاني
\InlineCode{openMode}.
في الواقع، هناك شفرة تدلّ للحاسوب على أنك تريد أن تفتح الملف بوضع القراءة فقط، الكتابة فقط أو الاثنين معاً.\\
هذه هي أوضاع فتح الملف المختلفة :
\begin{itemize}
  \item \textbf{\textenglish{"r"} :
قراءة فقط
(\textenglish{read only})}.
يمكنك قراءة محتوى الملف، و لكن لا يمكنك الكتابة فيه.
\textit{يجب أن يكون الملف موجوداً من قبل}.
  \item \textbf{\textenglish{"w"} :
كتابة فقط
(\textenglish{write only})}.
يمكنك الكتابة في الملف، لكن لا يمكنك قراءة محتواه.
\textit{إذا لم يكن الملف موجوداً من قبل، فإنه سيتم إنشاؤه}.
  \item \textbf{\textenglish{"a"} :
إلحاق
(\textenglish{append})}.
يمكنك الكتابة في الملف، إنطلاقا من نهايته.
\textit{إن لم يكن الملف موجوداً، فسيتم إنشاؤه}.
  \item \textbf{\textenglish{"r+"} :
قراءة و كتابة
(\textenglish{read and write})}.
يمكنك القراءة من الملف و الكتابة فيه.
\textit{يجب أن يكون الملف موجوداً من قبل}.
  \item \textbf{\textenglish{"w+"} :
قراءة و كتابة مع مسح المحتوى أوّلا}.
سيتم تفريغ الملف من محتواه أولاً، ثم بإمكانك الكتابة فيه و قراءة محتواه بعد ذلك.
\textit{إن لم يكن الملف موجوداً من قبل، سيتم إنشاؤه}.
  \item \textbf{\textenglish{"a+"}
إلحاق مع القراءة / الكتابة في آخر الملف}.
يمكنك القراءة و الكتابة إنطلاقا من نهاية الملف.
\textit{إن لم يكن موجوداً، سيتم إنشاؤه}.
\end{itemize}

لمعلوماتك، أنا عرضت لك بعضا من أوضاع فتح ملف. في الحقيقة، يوجد ضعفها !
من أجل كل وضع رأيناه هنا، إن أضفت
\InlineCode{"b"}
بعد المحرف الأول
(\InlineCode{"rb"}، \InlineCode{"wb"}، \InlineCode{"ab"}، \InlineCode{"rb+"}، \InlineCode{"wb+"}، \InlineCode{"ab+"})،
فإن الملف سيتم فتحه بالوضع الثنائي
(\textenglish{binary}).
هذا وضع خاص قليلاً فلن ندرسه هنا. في الواقع وضع النص يختصّ بتخزين... النص، تماما كما يوحي الاسم (فقط المحارف القابلة للعرض). أما الوضع الثنائي، يسمح بتخزين المعلومات
بايتا بايتا
(\textenglish{Byte by byte})
(أرقام بشكل أساسي). هذا مختلف كثيرا. على أي حال فطريقة العمل هي تقريبا نفس الّتي سنراها هنا.

شخصياً، أستعمل كثيراً الأوضاع :
\InlineCode{"r"}
(قراءة)،
\InlineCode{"w"}
(كتابة)،
\InlineCode{"r+"}
(قراءة و كتابة في آن واحد). وضع
\InlineCode{"w+"}
خطر قليلاً لأنه يقوم بمسح محتوى الملف مباشرة، بدون أن يطلب التأكيد قبل القيام بذلك. إن هذا الوضع ليس مفيداً إلا إذا أردنا أن نعيد تهيئة الملف أوّلا.
وضع الإلحاق
(\InlineCode{"a"})
يمكنه أن يفيد في بعض الحالات، إذا كنت تريد إضافة معلومات إلى نهاية الملف.

\begin{information}
  إن كنت تريد قراءة ملفّ، فمن المستحسن وضع
\InlineCode{"r"}.
بالطبع، الوضع
\InlineCode{"r+"}
يعمل أيضا، لكن بوضع
\InlineCode{"r"}
فأنت تضمن أنّ الملفّ لا يمكن تعديله، هذا نوع من الحماية.
\end{information}

إن كتبت دالةً
\InlineCode{loadLevel}
(لتحميل مستوى في لعبة مثلا)، الوضع
\InlineCode{"r"}
كافٍ، أما إن أردت أن كتابة دالةٍ
\InlineCode{saveLevel}
(لحفظ المستوى) فستستعمل الوضع
\InlineCode{"w"}.

الشفرة التالية ستفتح الملف
\InlineCode{test.txt}
في وضع
\InlineCode{"r+"}
(قراءة و كتابة) :

\begin{Csource}
int main(int argc, char *argv[])
{
	FILE* file = NULL;
	file = fopen("test.txt", "r+");
	return 0;
}
\end{Csource}

المؤشّر
\InlineCode{file}
يصبح إذن مؤشراً على الملف
\InlineCode{test.txt}.

\begin{question}
  أين يجب أن يكون الملف
\InlineCode{test.txt}
؟
\end{question}

يجب أن يكون في نفس المجلّد الذي يتواجد به الملف التنفيذي
(\InlineCode{.exe}).\\
من أجل متطلّبات هذا الفصل، أطلب منك أن تقوم بإنشاء ملف
\InlineCode{test.txt}
في نفس المساؤ الذي به
\InlineCode{.exe}،
مثلما أفعل أنا (الشكل الموالي).

\Picture{Chapter_II-7_Files}

كما ترى فأنا أستعمل  حاليّا بيئة التطوير
\textenglish{Code::Blocks}
الأمر الذي يفسّر وجود ملف المشروع بصيغة
\InlineCode{.cbp}
(في مكان الصيغة
\InlineCode{.sln}
إن كنت تستعمل
\InlineCode{Visual C++}
مثلاً). باختصار، الأمر المهم هو أن برنامجي
(\InlineCode{tests.exe})
موجود في نفس مجلّد الملف الذي نريد قرائته أو كتابته
(\InlineCode{test.txt}).

\begin{question}
  هل يجب أن يكون الملف بصيغة
\InlineCode{.txt} ؟
\end{question}

لا، الأمر يعود إليك في اختيار صيغة الملف عندما تفتحه. أي أنه بإمكانك أن تخترع صيغتك الخاصّة
\InlineCode{.level}
لحفظ مستويات ألعابك مثلاً.

\begin{question}
  هل من الواجب أن يكون الملف الذي نريد فتحه في نفس دليل الملف التنفيذي ؟
\end{question}

لا أيضا. يمكنه أن يكون داخل مجلّد بذات الدليل :

\begin{Csource}
  file = fopen("directory/test.txt", "r+");
\end{Csource}

هنا، الملف
\InlineCode{test.txt}
في مجلّد  داخليّ اسمه
\InlineCode{directory}.
هذه الطريقة التي نسميها
\textit{المسار النسبي}
عمليّة أكثر. هكذا، يمكن للبرنامج أن يعمل أينما كان مثبّتا.

من الممكن أيضا فتح ملفّ أينما كان في القرص الصلب. في هذه الحالة يجب كتابة المسار الكامل (ما نسميه
\textit{المسار المطلق}) :

\begin{Csource}
  file = fopen("C:\\Program Files\\Notepad++\\readme.txt", "r+");
\end{Csource}

هذه الشفرة تفتح الملف
\InlineCode{readme.txt}
الموجود بـ\InlineCode{C:\textbackslash Program Files\textbackslash Notepad++}.

\begin{warning}
  تعمّدت استعمال شرطتبن خلفيّتين
\InlineCode{\textbackslash}
  كما تلاحظ. في الواقع، إن كتبت اشارة واحدة، سيعتقد الحاسوب أنني أريد أن استخدم رمزا خاصا (مثل الـ\InlineCode{\textbackslash n}
أو الـ\InlineCode{\textbackslash t}).
لكتابة شرطة خلفيّة في سلسلة، يجب كتابتها إذن مرّتين ! هكذا يمكن أن يفهم أنّك تريد استخدام الرمز
\InlineCode{\textbackslash}.
\end{warning}

المشكل مع المسارات المطلقة، هو أنها لا تعمل إلا مع نظام معيّن، فهي ليست حلّا محمولا إذن. أي أنه لو كنت تعمل على
\textenglish{GNU/Linux}
لكان عليك كتابة مسار كهذا مثلا :

\begin{Csource}
  file = fopen("/home/mateo/directory/readme.txt", "r+");
\end{Csource}

لهذا فأنا أنصحك بكتابة مسارات نسبية. لا تستعمل المسارات المطلقة إلا في حالة كان البرنامج مخصص لنظام تشغيل معيّن، ليعدّل على ملف معيّن في القرص الصلب.

\subsection{اختبار فتح ملف}
المؤشّر
\InlineCode{file}
يجب أن يحوي عنوان الهيكل من نوع
\InlineCode{FILE}،
و الذي نستعمله كواصف
(\textenglish{descriptor})
للملف. هذا الواصف تم تحميله من أجلك في الذاكرة من طرف الدالة
\InlineCode{fopen}.
بعد هذا، هناك احتمالان :

\begin{itemize}
  \item إمّا أن تنجح عملية الفتح، فسنتمكن من المواصلة (أي البدء في القراءة و الكتابة في الملف).
  \item إمّا ألّا تنجح لأن الملف ليس موجوداً أو أنه مستخدم من طرف برنامج آخر. في هذه الحالة، سنتوقف عن العمل على الملف.
\end{itemize}

مباشرة بعد فتح الملف، يجب التأكد ما إن تمت العملية بنجاح، أم لا. هذا أمر بسيط : إذا كانت قيمة المؤشر تساوي
\InlineCode{NULL}،
فإن الفتح قد فشل. إن كانت قيمته تساوي شيئا غير
\InlineCode{NULL}،
فقد تم الفتح بنجاح.\\
سنتبع إذن هذا المخطط التالي :

\begin{Csource}
int main(int argc, char *argv[])
{
	FILE* file = NULL;
	file = fopen("test.txt", "r+");
	if (file != NULL)
	{
    		// We can read or write in the file
	}
	else
	{
    		// We display an error message if we want
    		printf("Can't open the file test.txt");
	}
	return 0;
}
\end{Csource}

إفعل هذا دائما عند فتح أي ملف. إن لم تفعل و الملف غير موجود، فأنت تخاطر بتوقّف البرنامج بعدها.

\subsection{\texttt{fclose} : غلق الملف}
إذا نجحت عملية فتح الملف، يمكننا القراءة و الكتابة فيه (سنرى كيف نفعل هذا لاحقاً).\\
ما إن نكمل العمل على الملف، يجب علينا "غلقه". نستعمل من أحل هذا الدالة
\InlineCode{fclose}
التي تقوم بتحرير الذاكرة. يعني أنّه سيتم حذف الملف المحمّل في الذاكرة العشوائية.

نموذج الدالة :

\begin{Csource}
int fclose(FILE* pointerOnFile);
\end{Csource}

هذه الدالة تأخذ معاملا واحدا : المؤشر نحو الملف.

تقوم بإرجاع
\InlineCode{int}،
و الذي يأخذ القيم :

\begin{itemize}
  \item \InlineCode{0} : إذا نجح غلق الملف.
  \item \InlineCode{EOF} : إذا فشل الغلق.
\InlineCode{EOF}
هي عبارة عن
\InlineCode{\#define}
موجودة في
\InlineCode{stdio.h}
و هي توافق عدداً خاصاً، يُستعمل للقول أنه حصل خطأ، أو أننا وصلنا إلى نهاية الملف. في حالتنا هذه، هذا يعني حدوث خطأ.
\end{itemize}

في غالب الأحيان، تنجح عملية غلق الملف : هذا ما يدفعني إلى عدم اختبار إن كانت
\InlineCode{fclose}
قد عملت. رغم هذا، يمكنك فعل ذلك إن أردت.

لإغلاق الملف، نكتب إذن :

\begin{Csource}
fclose(file);
\end{Csource}

في النهاية، المخطّط الذي نتّبعه لفتح و غلق ملف سيكون كالتالي :

\begin{Csource}
int main(int argc, char *argv[])
{
	FILE* file = NULL;
	file = fopen("test.txt", "r+");
	if (file != NULL)
	{
    		// We read and we write in the file
    		// ...
    		fclose(file); // We close the opened file
	}
	return 0;
}
\end{Csource}

لم استعمل
\InlineCode{else}
لأظهر رسالة خطأ في حال لم ينجح الفتح، يمكنك فعل ذلك إن أردت.

يجب دائما التفكير في غلق الملف الذي فتحته بمجرّد الإنتهاء من العمل عليه. هذا سيسمح بتحرير الذاكرة.\\
إن نسيت تحرير الذاكرة، قد يأخذ برنامجك حجما كبيراً من الذاكرة بدون أن يستخدمه. في مثال صغير كهذا الأمر غير خطير، لكن مع برنامج كبير، مرحباً بالمشاكل !

نسيان تحرير الذاكرة أمر يقع. بل سيحدث لك هذا كثيرا. في هذه الحالة نقول أنّه قد حدث
\textit{تسريب للذاكرة}.
هذا يجعل برنامجك يستخدم قدرا من الذاكرة أكبر من اللازم بدون أن تفهم سبب حصول ذلك. في غالب الأحيان، يكون السبب واحدا أو إثنين من الأمور "الثانوية" مثل نسيان
\InlineCode{fclose}.

\section{طرق مختلفة للقراءة و الكتابة في الملفات}
و الآن مادمنا تعلّمنا كيف نفتح و نغلق ملفا، لم يبق سوى أن نضيف الشفرة الّتي تقوم بالقراءة و الكتابة عليه.

سنبدأ برؤية كيفيّة الكتابة في ملفّ (الأمر الأبسط قليلا)، ثمّ نمرّ يعدها إلى كيفيّة القراءة من ملفّ.
