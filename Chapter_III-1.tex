\chapter{تثبيت الـ\textenglish{SDL}}

ابتداءاً من الآن، إنتهت الدروس النظرية ! لأننا سنمرّ إلى مرحلة مهمّة، و سنستمتع بالتطبيق بالإستعانة بمكتبة نسميها
\underline{\textenglish{SDL}}.

في الدروس السابقة كنا قد تطرّقنا تقريباً لكلّ أساسيات اللغة
\textenglish{C}،
لكن تبقى هناك دائماً بعض التفاصيل الصعبة نوعاً ما لنكتشفها. سأقول لك بأنه يُمكن لهذا الكتاب أن يتوقّف هنا مخبرا إيّاك : "نعم لقد تعلّمت البرمجة بلغة 
\textenglish{C}"،
لكني متأكّد بأن الجميع سيشاركني الرأي لو قلت بأن المُبرمج سيحسّ نفسه دائماً مبتدئاً مادام لم "يخرج" من الكونسول !

الـ\textenglish{SDL}
هي مكتبة تُستخدم خاصّة لإنشاء ألعاب ثنائية الأبعاد. سنتعرّف في هذا الدرس على هذه المكتبة و نتعلّم كيف نقوم بتثبيتها.

نسمي هذا النوع من المكتبات بمكتبات الطرف الثالث 
(\textenglish{third party libraries}).
يجب أن تعرف أنه هناك نوعين من المكتبات :

\begin{itemize}
	\item \textbf{المكتبة القياسية}
	(\textenglish{standard library}) :
	و هي المكتبة القاعدية التي تعمل على كلّ أنظمة التشغيل (من هنا تم استنباط الكلمة 
	\textenglish{standard})
	و هي تسمح بالقيام بأمور بسيطة كـ\InlineCode{printf}.
	هذه المكتبات يتمّ تسطيبها تلقائيّا عند تثبيتك للبيئة التطويرية و المترجم.
	
	خلال الجزئين الأوّلين من هذا الكتاب، كناّ قد استعملنا المكتبة القياسيّة فقط
(\InlineCode{stdlib.h}، \InlineCode{stdio.h}، \InlineCode{string.h}، \InlineCode{time.h} \dots).
	لم نقم بدراستها بالتفصيل لكنّا جرّبنا منها جزءاً كبيراً. إن كنت تريد معرفة المزيد عن هذا النوع من المكتبات أجْرِ بحثاً في 
	\textenglish{Google}،
	مثلاً بكتابة
	"\textenglish{C standard library}"،
	و ستجد نماذج الدوال في هذه المكتبة، بالإضافة إلى شرح قصير حول دور كلّ دالة.
	\item \textbf{مكتبات الطرف الثالث}
	(\textenglish{third party libraries}):
	هي مكتبات لا يتم تثبيتها تلقائيا. و إنّما يجب عليك تنزيلها من الأنترنت و تثبيتها بنفسك على حاسوبك.

	على عكس المكتبات القياسية، التي تكون بسيطة نسبيّا و تحتوي على عدد قليل من الدوال، فإنه توجد الآلاف من مكتبات الطرف الثالث، و التي تمت كتابتها من طرف مبرمجين آخرين. بعضها جيّدة، و أخرى أقل، بعضها مدفوع، و بعضها الآخر مجاني، إلخ. الأمر المثالي هو إيجاد مكتبة جيّدة و مجانية في نفس الوقت !
\end{itemize}

إنه لمن المستحيل أن أضع لك درساً يشرح كل المكتبات الموجودة. حتّى لو أمضيت حياتي كلّها 24 ساعة / 24، لن أستطيع !\\
لذا سأقدّم لك مكتبة واحدة فقط مكتوبة بالـ\textenglish{C} و مُستعملة من طرف مبرمجين مثلك. 

هذه المكتبة تدعى 
\textit{\textenglish{SDL}}.
السؤال المطروح هو لماذا اخترت هذه المكتبة بالضبط ؟ ما الذي يميّزها عن باقي المكتبات ؟\\
هذه أسئلة سأبدأ في الإجابة عليها إنطلاقاً من الآن.
