\chapter{الحجز الحيّ للذاكرة
(\textenglish{Dynamic memory allocation})}

كل المتغيّرات التي أنشأناها لحد الآن تمّ إنشاؤها تلقائيّا من طرف المترجم الخاصّ بلغة
\textenglish{C}.
لقد كانت الطريقة البسيطة. رغم ذلك، توجد طريقة يدوية أكثر لإنشاء متغيّرات و نسمّيها بالحجز الحيّ
(\textenglish{Dynamic allocation}).

من بين فوائد الحجز الحيّ هو السماح لبرنامج بحجز مكان لازم لتخزين جدول في الذاكرة لا يُعرف حجمه قبل بداية الترجمة. في الواقع، حتّى الآن، كان حجم جداولنا ثابتاً في الشفرة المصدريّة. بعد قراءة هذا الفصل، ستستطيع إنشاء جداول بطريقة أكثر مرونة !

من الضروري أن تتقن التعامل مع المؤشرات لتتمكّن من قراءة هذا الفصل ! إن كانت لديك بعض الشكوك حول المؤشرات، أنصحك بالذهاب لإعادة قراءة الفصل الموافق قبل البدأ.

عندما نقوم بالتصريح عن متغيّر، فإننا نقول أننا
\textbf{طلبنا حجز مكان في الذاكرة} :

\begin{Csource}
int myNumber = 0;
\end{Csource}

عندما يصل المترجم إلى سطر مشابه للسطر السابق، يقوم بالأمور التالية :
\begin{itemize}
  \item يقوم البرنامج بطلب إذن من نظام التشغيل
(\textenglish{Windows}، \textenglish{GNU/Linux}، \textenglish{Mac OS}...)
ليحجز شيئا من الذاكرة.
  \item يستجيب نظام التشغيل بإعطاء البرنامج عنوان الخانة حيث يمكنه تخزين المتغيّر (يعطيه العنوان الّذي حجزه له).
  \item عندما تنتهي الدالّة، المتغيّر يتم حذفه من الذاكرة. برنامجك يقول لنظام التشغيل : "أنا لم أعد بحاجة إلى المكان في الذاكرة الّذي حجزته في ذلك العنوان، شكرا ! التاريخ لا يحدّد إن كان البرنامج قد قال فعلا "شكرا" لنظام التشغيل، لكنّ هذا في مصلحته لأنّ نظام التشغيل هو الّذي يتحكم في الذاكرة !
\end{itemize}

لحد الآن كل الأمور كانت تلقائيّة. عندما نصرّح عن متغير فإن نظام التشغيل يتمّ استدعاءه تلقائياً من طرف البرنامج.
ما رأيك إذا بفعل هذا بطريقة يدوية ؟ ليس لأننا نريد أن نستمتع بفعل شيء معقّد، بل لأننا أحيانا نظطرّ لفعل ذلك !

في هذا الفصل سنقوم بـ :
\begin{itemize}
  \item دراسة كيف تعمل الذاكرة (نعم، مرّة أخرى !) لنعرف ما الحجم الذي يحجزه كل متغيّر حسب نوعه.
  \item ثمّ ندخل في موضوعنا الأساسي : سنرى كيف نطلب من نظام التشغيل يدويّا أن يحجز لنا مكانا في الذاكرة. هذا ما سنسميه الحجز الحيّ للذاكرة.
  \item و أخيراً، سنكتشف الفائدة من القيام بالحجز الحيّ بتعلّم إنشاء جدول ذو حجم غير معروف إلّا عند اشتغال البرنامج.
\end{itemize}
