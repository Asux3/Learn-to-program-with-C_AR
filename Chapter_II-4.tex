\chapter{السلاسل المحرفيّة}
السلاسل المحرفيّة (\textenglish{Strings}) هي اسم صحيح
\textit{برمجيّا}
لتسمية ... النصّ، ببساطة !\\
السلسلة المحرفيّة هي إذن نصّ يمكننا حفظه على شكل متغيّر في الذاكرة. بهذه الطريقة يمكننا تخزين اسم المستخدم.

كنت قد قلت من قبل أن الحاسوب لا يفهم إلا الأعداد، فما هو السحر الذي يفعله المبرمجون للتعامل مع النصوص ؟ إنّعم ماكرون، سوف ترى !

\section{النوع \texttt{char}}
في هذا الفصل سنعطي أهمية خاصة للنوع
\InlineCode{char}.
إن كنت تتذكّر جيداً فهذا النوع يسمح بخزين الأعداد المحصورة بين
$-127$
و
$128$.

\begin{information}
  النوع
\InlineCode{char}
 يسمح بتخزين الأعداد، لكننا غالبا لا نستخدمه في لغة الـ\textenglish{C}
من أجل ذلك.
عادة، حتّى لو كان العدد صغيرا، فإنّنا نخزنه في
\InlineCode{int}.
بالطبع، هذا سيأخذ  شيئا أكبر من الذاكرة، لكن في هذه الأيّام، ليست الذاكرة ما ينقص الحواسيب فعلا.
\end{information}
إذا فالنوع
\InlineCode{char}
مستعمل لتخزين ... "حرف" ! إحذر، لقد قلت :
\underline{حرف واحد}.

و لأنّ الذاكرة لا يمكنها تخزين شيء سوى الأعداد، فلقد تمّ اختراع جدول يقوم بالتحويل بين الحروف و الأعداد. هذا الجدول يخبرنا مثلا أنّ العدد 65 مكافئ للحرف
\textenglish{A}.

لغة
\textenglish{C}
تسمح لنا بالقيام بالتحويل بين الحرف و العدد الموافق له. للحصول على العدد الموافق لحرف، يكفي كتابته بين علامتي تنصيص، هكذا :
\InlineCode{'A'}.
عند الترجمة، سيتم استبدال
\InlineCode{'A'}
بالقيمة الموافقة.

فلنجرّب :
\begin{Csource}
int main(int argc, char *argv[])
{
	char letter = 'A';
	printf("%d\n", letter );
	return 0;
}
\end{Csource}
نعلم إذن أن الحرف
\textenglish{A}
يمثل بالعدد 65،
\textenglish{B}
بـ66،
\textenglish{C}
بـ67، إلخ.\\
جرّبوا بالأحرف الصغيرة و سترون أن القيم مختلفة. في الواقع، الحرف
\InlineCode{'a'}
ليس مطابقا لـ\InlineCode{'A'}،
الحاسوب يقوم بالتفريق بين الحروف الصغيرة و الكبيرة (نقول أنّه يحترم حالة الحرف).

أغلب الحروف "الأساسيّة" مشفّرة بين $0$ و $127$. يوجد جدول يقوم بالتحويل بين الأعداد و الحروف : الجدول
\textenglish{ASCII}
(ينطق "أسكي"). الموقع
\href{http://www.asciitable.com/}{AsciiTable.com}
مشهور لعرض هذا الجدول لكنّه ليس الوحيد، يمكننا أن نجده على ويكيبيديا و مواقع أخرى أيضا.

\subsection{إظهار محرف}
كما نعلم فلإظهار أي شيء على الشاشة نستعمل الدالة
\InlineCode{printf}،
هذه الدالة قادرة أيضا على إظهار محرف على الشاشة و ذلك باستعمال الرمز :
\InlineCode{\%c}
(\textenglish{c} تعني \textenglish{character})
كالتالي :
\begin{Csource}
int main(int argc, char *argv[])
{
	char letter = 'A';
	printf("%c\n", letter);
	return 0;
}
\end{Csource}
\begin{Console}
A
\end{Console}
حسنا، لقد تعلّمنا كيف نظهر حرفا في الشاشة.

يمكننا أيضا أن نطلب من المستعمل أن يقوم بإدخال حرف عن طريق لوحة المفاتيح، و ذلك بالإستعانة بالدالة
\InlineCode{scanf}،
و هذا بوضع الرمز
\InlineCode{\%c}
دائماً، كالتالي :
\begin{Csource}
int main(int argc, char *argv[])
{
	char letter = 0;
	scanf("%c", &letter);
	printf("%c\n", letter);
	return 0;
}
\end{Csource}
إن كتبت الحرف
\textenglish{B}
فسأتحصّل على :
\begin{Console}
B
B
\end{Console}
أوّل
\textenglish{B}
هو الّذي كتبته، أمّا الثاني فهو المعروض من طرف
\InlineCode{printf}.

هذا تقريبا ما يجب عليك أن تتذكره بخصوص النوع \InlineCode{char}.
تذكّر جيّدا :
\begin{itemize}
  \item  النوع
\InlineCode{char}
يخزّن الأعداد من
$-128$
إلى
$127$،
بينما النوع
\InlineCode{unsigned char}
يسمح بتخزين الأعداد من
$0$
إلى
$255$.
  \item يستخدم الحاسوب جدولا للتحويل بين الحروف و الأعداد، الجدول
\textenglish{ASCII}.
  \item يمكن استخدام
\InlineCode{char}
لتخزين حرف
\underline{واحد}.
  \item \InlineCode{'A'}
يتمّ استبدالها أثناء الترجمة بالقيمة الموافقة (65 مثلا). نستخدم إذن علامات التنصيص للحصول على قيمة حرف.
\end{itemize}
