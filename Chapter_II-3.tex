\chapter{الجداول}
هذا الدرس هو ملحق مباشر للدرس المتعلق بالمؤشرات، و سيعلّمك أهميتها أكثر. إن كنت تعتقد بأنك قادر على تفادي المؤشرات فأنت مخطئ ! هي في كلّ مكان في لغة الـ\textenglish{C}. لقد حذّرتك !

سنتعلم في هذا الدرس كيف ننشئ متغيرات من نوع "جداول". الجدوال مهمّة للغاية في لغة الـ\textenglish{C} لأنها تساعد في تنظيم سلسلة من القيم.

نبدأ هذا الدرس ببعض الشروحات و التفسيرات حول كيفية عمل الجداول في الذاكرة (سأقدم لك الكثير من المخططات التفسيرية). هذه المقدمات حول الذاكرة مهمة جداً : ستساعدك في في معرفة عمل الجداول. فمن المستحسن أن يعرف المبرمج ما يقوم به كي يتحكم في برامجه أكثر، أليس كذلك ؟

\section{الجداول في الذاكرة}
\textit{"الجداول هي تتابع متغيرات من نفس النوع، موجودة في مكان متواصل من الذاكرة."}

أعرف أن هذا التعريف يشبه قليلا تعريف القاموس. لهذا فسأوضح بطريقة أخرى، فعلياّّ، الجدول عبارة عن "متغيّرات ضخمة" يمكن لها أن تحتوي على أعداد كبيرة من نفس النوع
(\InlineCode{char}،
\InlineCode{long}،
\InlineCode{int}،
\InlineCode{double}...).

للجدول طول محدد. يمكنه أن يكون 2، 3، 10 خانات، 150، 2500 خانة، أنت من يحدد العدد. المخطط التالي مثال عن جدول يحجز 4 خانات بدءاً بالعنوان 1600 :
\Picture{Chapter_II-3_Array-Adresses}
عندما تطلب إنشاء جدول يحجز 4 خانات في الذاكرة، سيطلب برنامجك من نظام التشغيل أن يسمح له باستغلال 4 خانات في الذاكرة، و يجب ان تكون هذه الخانات متتالية يعني الواحدة بجانب الأخرى. و كما ترى أعلاه فالخانات متتابعة 1600, 1601, 1602, 1603 فلا يوجد "فراغ" بينها.

أخيراً، كل خانة تحتوي عددا من نفس النوع. فإن كان الجدول من نوع
\InlineCode{int}
فإن كلّ خانة يجب أن تحتوي عددا من نوع
\InlineCode{int}.
و بهذا نفهم أنه لا يمكننا وضع نوع
\InlineCode{int}
مع
\InlineCode{double}
في الجدول نفسه.

و كتلخيص، هذا أهم ما يجب أن تعرفه بخصوص الجداول :
\begin{itemize}
  \item عندما يتم إنشاء جدول، يأخذ مكانا متواصلاً في الذاكرة. بحيث تكون الخانات متجاورة الواحدة تلو الأخرى.
  \item كل خانات الجدول تكون من نفس النوع، فجدول
الـ\InlineCode{int}
يمكن أن يحمل فقط
\InlineCode{int}،
و لا أي نوع آخر.
\end{itemize}

\section{تعريف جدول}
كي نبدأ سننشئ جدولا من 4 أعداد من نوع
\InlineCode{int} :
\begin{Csource}
int table[4];
\end{Csource}
هذا كلّ شيء. يكفي إذن أن تضيف قوسين مربعين
(\InlineCode{[}
و
\InlineCode{]})
عدد الخانات التي تريد أن يحجزها جدولك، و اعلموا أنه لا يوجد حدود (إلا إن تجاوزت الحدّ الذي تسمح به ذاكرة جهازك طبعا).

و لكن الآن، كيف نصل لخانة ما في الجدول ؟
هذا سهل، تكفي كتابة
\InlineCode{table[cellNumber]}.
\begin{critical}
  إحذر : كل جدول يجب أن يبدأ بالفهرس رقم 0 ! جدولنا متكوّن من 4
\InlineCode{int}
إذن فالفهارس المتوفرة هي : 0، 1، 2 و 3. لا وجود للفهرس 4 في جدول من 4 خانات ! هذا مصدر أخطاء متداولة، فلا تغفل عنه !
\end{critical}
إذا كنت أريد أن أضع في جدولي نفس القيم التي في المخطط فبجب إذا أن أكتُب :
\begin{Csource}
int table[4];
table[0] = 10;
table[1] = 23;
table[2] = 505;
table[3] = 8;
\end{Csource}
\begin{question}
  لازلت لا أرى العلاقة بين المؤشرات و الجداول ؟
\end{question}
في الواقع، لو تكتب فقط
\InlineCode{table}
فستحصل على مؤشر، و هو مؤشر على الخانة الأولى من الجدول، قم باختبار التالي :
\begin{Csource}
int table[4];
printf("%d", table);
\end{Csource}
النتيجة ستظهر لك العنوان الذي يتواجد به
\InlineCode{table} :
\begin{Console}
1600
\end{Console}
بينما إذا قمت بوضع فهرس الخانة بين قوسين مربعين، فستحصل على القيمة :
\begin{Csource}
int table[4];
printf("%d", table[0]);
\end{Csource}
\begin{Console}
10
\end{Console}
نفس الشيء بالنسبة للفهارس الأخرى. بما أن
\InlineCode{table}
هو مؤشر، يمكننا استعمال الرمز
\InlineCode{*}
للحصول على القيمة الأولى :
\begin{Csource}
int table[4];
printf("%d", *table);
\end{Csource}
\begin{Console}
10
\end{Console}
يمكن أيضا الحصول على قيمة الخانة الثانية بكتابة
\InlineCode{*(table + 1)}
(أي عنوان الجدول + 1). لذا فهذان السطران متماثلان :
\begin{Csource}
table[1] // Returns the value of the second cell (the first is 0)
*(table + 1) // Same thing : returns the value of the second cell.
\end{Csource}
لذا فعند كتابة
\InlineCode{table[0]}،
فأنت تطلب قيمة الخانة التي تتواجد بعنوان الجدول + 0
خانة، أي 1600.\\
و إذا كتبت
\InlineCode{table[1]}
فإنك تطلب القيمة المتواجدة في عنوان الجدول + 1
خانة، أي 1601.\\
و هكذا من أجل الباقي.

\subsection{الجداول ذات الحجم المتغيّر}
هناك عدة نسخ من لغة
\textenglish{C}.\\
نسخة حديثة منها تدعى
\textenglish{C99}
تسمح بإنشاء جداول ذات حجم متغيّر. يعني أن حجم الجداول يمكن أن يكون معرّفا بمتغير.
\begin{Csource}
int size = 5;
int table[size];
\end{Csource}
إلا أن هذه الكتابة ليست مفهومة بالنسبة لكل الـمترجمات
(\textenglish{compilers})
فبعضها تتوقف في السطر الثاني. إن لغة
الـ\textenglish{C}
الّتي اعلمك إياها منذ البداية (تدعى
\textenglish{C89})
لا تسمح بهذا النوع من الكتابات. و لذا يمكننا القول أن فعل هذه الأشياء أمر ممنوع.

يجب أن نتفق على شيء و هو : لا تملك الحق في وضع متغير بين القوسين المربعين من أجل تعريف حجم الجدول. حتى و إن كان المتغير ثابتا ! يجب على طول الجدول أن يأخذ قيمة ثابتة، و لهذا عليك أن تحدده كعدد :
\begin{Csource}
int table[5];
\end{Csource}
\begin{question}
   إذن...  هل من الممنوع إنشاء جدول يعتمد حجمه على قيمة متغير ؟
\end{question}
بلى إنه ممكن حتى مع
\textenglish{C89}.
لكن لفعل هذا سنعتمد على تقنية أخرى (أكيدة أكثر و تعمل مع كل المترجمات) تدعى بـ\textbf{الحجز الحيّ}
(\textenglish{Dynamic allocation}).
سندرسها في مرحلة متقدمة من هذه الدروس.

\section{تصفح جدول}
لنفرض أنني أريد الآن أن أُظهر كل قيم خانات الجدول.\\
يمكنني أن أستدعي الدالة
\InlineCode{printf}
بالقدر الذي يحتويه الجدول من خانات. لكن سيكون الأمر ثقيلاً و مليئاً بالتكرار، و تخيل حجم الشفرة المصدرية لو أننا أردنا إظهار قيم الجدول واحدة بواحدة !

الأحسن هو أن نستعين بحلقة. لم لا حلقة
\InlineCode{for}
؟ فهي الأنسب لتصفح الجداول :
\begin{Csource}
int main(int argc, char *argv[])
{
	int table[4], i = 0;
	table[0] = 10;
	table[1] = 23;
	table[2] = 505;
	table[3] = 8;
	for (i = 0 ; i < 4 ; i++)
	{
    		printf("%d\n", table[i]);
	}
	return 0;
}
\end{Csource}
\begin{Console}
10
23
505
8
\end{Console}
إن حلقتنا تتصفح الجدول بمساعدة متغير يسمى
\InlineCode{i}
(إسم شائع لدى المبرمجين يخص المتغير الذي يستخدم لتصفح جدول !).

إن الشيء العَمَلِيَّ خاصة، هو أنه بإمكاننا وضع متغير داخل قوسين مربعين بالفعل. فالمتغير كان ممنوعا في مرحلة إنشاء الجدول (لتعريف حجمه). لكن و لحسن الحظ، فهي مسموحة من أجل "تصفح" الجدول، أي إظهار قيمه !\\
هنا قد نعطي المتغير
\InlineCode{i}
بشكل متتالي القيم 0، 1، 2، 3. بهذا سنقوم إذن بإظهار قيمة
\InlineCode{table[0]}،
\InlineCode{table[1]}،
\InlineCode{table[2]}
و
\InlineCode{table[3]} !
\begin{critical}
  إنتبه لعدم محاولة إظهار قيمة
\InlineCode{table[4]} !
فجدول من 4 خانات يتضمن الفهارس 0، 1، 2، 3 فقط. فإن حاولت عرض
\InlineCode{table[4]}
فإمّا إن تحصل على قيمة عشوائية، أو سيظهر لك خطأ جميل، نظام التشغيل يوقف برنامجك فهو يحاول الوصول لعنوان لا ينتمي إليه.
\end{critical}

\subsection{تهيئة جدول}
الآن و مادمنا قد عرفنا كيف نتصفح جدولا، يمكننا أن نضبط كل قيمه على 0 باستخدام حلقة !

إن القيام بتصفح جدول لضبط كلّ قيمه على الصفر أمر يمكن القيام به بمستواك هذا :
\begin{Csource}
int main(int argc, char *argv[])
{
	int table[4], i = 0;
	// Initialization of the table
	for (i = 0 ; i < 4 ; i++)
	{
    		table[i] = 0;
	}
	// Displaying the values of the table to check
	for (i = 0 ; i < 4 ; i++)
	{
    		printf("%d\n", table[i]);
	}
	return 0;
}
\end{Csource}
\begin{Console}
0
0
0
0
\end{Console}
