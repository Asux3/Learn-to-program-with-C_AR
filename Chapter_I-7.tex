\chapter{الحلقات التكرارية}

بعدما تعلمنا كيف ننشؤ شروطا بلغة 
\textenglish{C}،
سنكتشف معاً 
\underline{الحلقات التكراريّة} (\textenglish{loops}).
ما هي الحلقة ؟ هي تقنية تسمح بتكرار نفس التعليمات عدة مرات. و ستساعدنا كثيرا من الآن و صاعدا خاصة في العمل التطبيقي الأوّل الذي ينتظرنا بعد هذا الفصل.

استرخ : هذا الفصل سيكون سهلاً. لقد تعرفنا سابقا على ما تعنيه المتغيّرات المنطقية
(\textenglish{booleans})
و الشروط 
(\textenglish{conditions})
في الفصل السابق، و بذلك كنا قد تخلصنا من عمل كبير. من الآن فصاعداً ستكون الأمور سلسة أكثر و لن يكون في العمل التطبيقي القادم الكثير من المشاكل.

فلننتهز الفرصة، لأننا لن نتأخر في الدخول في الجزء الثاني من الكتاب. سيكون من الجيّد لك أن تنتبه !

\section{ماهي الـحلقة ؟}

كما قلت سابقاً : هي عبارة عن تعليمة تسمح لنا بتكرار نفس التعليمات عدة مرات. 

تماما مثل الشروط، توجد طرق عديدة لإنشاء الحلقات. و لكن مهما اختلفت الطرائق فالهدف واحد : تكرار تعليمات لعدد معيّن من المرات. \\
لدينا في لغة 
\textenglish{C}
ثلاثة أنواع من الحلقات :
\begin{itemize}
	\item \InlineCode{while}
	\item \InlineCode{do \dots while}
	\item \InlineCode{for}
\end{itemize}
في جميع الحالات يبقى المخطط نفسه :

\Picture{Chapter_I-7_Loop}
و هذا ما سيحصل بالترتيب :

\begin{enumerate}
	\item الجهاز يقرأ التعليمات من الأعلى إلى الأسفل كالعادة.
	\item ما إن يصل لنهاية الحلقة يتوجه نحو التعليمة الأولى.
	\item يعيد بعدها قراءة التعليمات كلها من الأعلى إلى الأسفل.
	\item يصل لنهاية الحلقة و يعاود الرجوع للأول من جديد و هكذا \dots
\end{enumerate}
4- 
المشكلة في هذا النظام هو أننا إن لم نقم بإيقافه، فالجهاز قادر على تكرار نفس التعليمات إلى مالانهاية ! و لن يتذمّر، أنت تعرف : هو يفعل ما تأمره أنت بفعله \dots يمكنه أن يعلق في حلقة غير منتهية، و هذا النوع من الحالات يعتبر مصدر خوف  بالنسبة للمبرمجين.

و هنا نجد \dots الشروط ! فعندما ننشئ حلقة نقوم دائما بتعريف شرطها. هذا الشرط يعني "كرّر الحلقة دون توقف مادام هذا الشرط صحيحا".

كما قلت فهناك عدة طرق للقيام بذلك و سنبدأ من دون تأخير بإنشاء حلقة من نوع 
\InlineCode{while}
في الـ\textenglish{C}.
