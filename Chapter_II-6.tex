\chapter{أنشئ أنواع متغيرات خاصّة بك}
تسمح لغة الـ\textenglish{C}
بالقيام بشيء يعتبر قوياً جداً : و هو أن ننشئ أنواعاً خاصة بنا، "أنواع متغيّرات مخصّصة". سنرى نمطين : الـهياكل
(\textenglish{Structures})
و التعدادات
(\textenglish{Enumerations}).

 إن إنشاء أنواع خاصّة بنا يعتبر أمراً ضروريا خاصة إذا أردنا إنشاء برامج أكثر تعقيداً.

الأمر ليس  (لحسن الحظّ) بالصعب، لكن ركّز جيّدا لأننا سنستعمل الهياكل كل الوقت انطلاقا من الفصل القادم.\\
يجب أن تعلم أنّ المكتبات تنشئ غالبا أنواعها الخاصّة. لن يمرّ وقت كثير حتّى تستخدم نوعا يدعى "ملف"، و بعده بقليل، أنواع أخرى مثل "نافذة"، "صوت"، "لوحة مفاتيح"، إلخ.

\section{تعريف هيكل}
الهيكل هو تجميع لعدد من المتغيرات التي يمكن لها أن تحمل أنواعا مختلفة. على عكس الجداول التي ترغمنا على استعمال خانات من نفس النوع في كلّ الجدول، بإمكانك تعريف هيكل يحمل الأنواع :
\InlineCode{long}، \InlineCode{char}، \InlineCode{int}
و
\InlineCode{double}
في مرّة واحدة.

الهياكل في أغلب الأحيان معرّفة في ملفات
\InlineCode{.h}
مثلما رأينا مع
\InlineCode{\#define}
و نماذج الدوال. هذا مثال عن هيكل :
\begin{Csource}
struct StructureName
{
	int variable1;
	int variable2;
	int anotherVariable;
	double decimalNumber;
};
\end{Csource}
لتعريف هيكل، يجب علينا أن نبدأ بالكلمة المفتاحية
\InlineCode{struct}،
متبوعة باسم الهيكل (مثلا
\InlineCode{File}
أو
\InlineCode{Screen}).

\begin{information}
  شخصيّا لديّ عادة في تسمية هياكلي بنفس قواعد تسمية المتغيّرات، باستثناء أنّي أجعل أوّل حرف كبيرا للتفريق. هكذا، عندما أرى الكلمة
\InlineCode{captainAge}
في شفرتي، أعلم أنّها متغيّر لأنّها تبدأ بحرف صغير. عندما أرى
\InlineCode{AudioPart}
فأعلم أنّها هيكل (نوع مخصّص) لأنّها تبدأ بحرف كبير.
\end{information}

بعد ذلك، نفتح حاضنة لنغلقها لاحقاً تماما مثل الدوال.

\begin{critical}
  احذر، الأمر خاصّ هنا : بالنسبة للهياكل، يجب أن تضع بعد الحاضنة النهائية فاصلة منقوطة. هذا أمر إجباري. إن لم تفعله فستتوقّف الترجمة.
\end{critical}
و الآن، ماذا نضع داخل الحاضنتين ؟\\
هذا سهل، سنضع المتغيرات التي يتكون منها الهيكل، و عادة ما يتكون الهيكل من "مُتَغَيِّريْن داخِلِيَيْن" على الأقل، و إلا فلن يحمل معنى كبيرا.

كما ترى، فإنشاء نوع متغيّرات مخصّص ليس بالأمر الصعب. كلّ الهياكل ماهي إلّا "تجميعات" لمتغيّرات من أنواع قاعديّة مثل
\InlineCode{long}، \InlineCode{int}، \InlineCode{double}،
إلخ. لا توجد معجزة، إنّ نوعا
\InlineCode{File}
مثلاً ما هو إلا مجموعة من الأعداد القاعديّة !

\subsection{مثال عن هيكل}
تخيل أنك تريد إنشاء متغيّر لكي يٌخزّن إحداثيات نقطة في معلم الشاشة. ستحتاج بالتأكيد إلى هيكل كهذا عندما تبدأ في برمجة ألعاب ثنائية الأبعاد في الجزء التالي من الكتاب، هذه إذن فرصة للتقدّم قليلا.

إذا كانت كلمة "علم الهندسة" تُحدث ظهوربقع غير مفهومة على كامل وجهك، فالمخطّط التالي سيذكّرك قليلا بأساسيّات الأبعاد الثنائيّة (\textenglish{2D}).
\Picture{Chapter_II-6_Axis}
عندما نعمل في
\textenglish{2D}
فلدينا محوران : محور الفواصل (من اليسار إلى اليمين) و محور التراتيب (من الأسفل إلى الأعلى). فمن العادة أن نرمز للفواصل بمتغيّر يدعى
\InlineCode{x}
و للتراتبب بـ\InlineCode{y}.

هل يمكنك كتابة هيكل
\InlineCode{Coordinates}
يسمح بتخزين كلّا من الفاصلة
(\InlineCode{x})
و الترتيبة
(\InlineCode{y})
لنقطة ما ؟\\
هيّا، هيّا، الأمر ليس صعبا :
\begin{Csource}
struct Coordinates
{
	int x; // Abscissas
	int y; // Ordinates
};
\end{Csource}
هيكلنا يسمّى
\InlineCode{Coordinates}
و هو متكوّن من متغيرين
\InlineCode{x}
و
\InlineCode{y}
أي الفاصلة
(\textenglish{Abscissa})
و الترتيبة
(\textenglish{Ordinate}).

إن أردنا، يمكننا بسهولة إنشاء هيكل
\InlineCode{Coordinates}
من أجل
\textenglish{3D}:
يكفي فقط إضافة متغيّر ثالث (مثلا
\InlineCode{z})
يدلّ على الارتفاع. بهذا سيكون لدينا هيكل لإدارة النقاط الثلاثيّة الأبعاد في الفضاء !
