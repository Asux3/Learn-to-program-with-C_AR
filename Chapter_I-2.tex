\chapter{الحصول على الأدوات اللازمة}

بعد تجاوزنا لدرس تمهيدي مليئ بالثرثرة سوف نبدأ بالدخول في صلب الموضوع. سوف نجيب عن السؤال التالي : "ما هي البرامج التي نحتاج إليها للبدء في البرمجة ؟".

لا يوجد شيء صعب في هذا الدرس، سوف نأخذ وقتنا للتأقلم على هذه البرامج الجديدة.

اغتنموا الفرصة ! في الفصل التالي سنبدأ حقّا في البرمجة و لن يكون هناك وقت للقيلولة !

\section{الأدوات اللازمة للمبرمج}

إذن ما هي الأدوات التي نحتاج إليها ؟
إذا تابعت الفصل السابق جيّدا، فستعرف واحدا على الأقل !

هل تعلم عمّا أتحدّث ؟ حقّا لا ؟

حسنا، نحن نتحدّث عن
\textbf{المترجم}
الذي يمكّن من ترجمة لغة الـ\textenglish{C}
إلى اللغة الثنائيّة !

كما قلت لكم في الفصل الأوّل، يوجد العديد من المترجمات للغة الـ\textenglish{C}.
سنرى أن اختيار المترجم ليس أمرا معقّدا في حالتنا هذه.

ما الذي نحتاج إليه أيضا ؟ لن أتركك تخمّن كثيرا و سأعطيك القائمة :

\begin{itemize}
  \item \textbf{محرّر نصوص }
(\textenglish{Text Editor})
لكتابة الشفرة المصدرية الخاصّة بالبرنامح. نظريّا برنامج تحرير نصوص بسيط مثل
\textenglish{Notepad}
على
\textenglish{Windows}
أو
\textenglish{vi}
على
\textenglish{Unix}
يكفي، لكن من الأحسن استخدام محرّر نصوص ذكيّ يقوم بتلوين الشفرة المصدرية لكي يسهّل عليك العمل.
  \item \textbf{مترجم}
  لتحويل الشفرة المصدرية إلى ملف ثنائي.
  \item \textbf{المنقّح}
(\textenglish{Debugger})
لمساعدك على كشف الأخطاء في برنامجك. لسوء الحظ، لم نتمكّن بعد من ابتكار "المصحّح" الّذي يصحّح أخطائك لوحده. لكن، إن أحسنت استخدام المنقّح، يمكنك ببساطة إيجاد الأخطاء.
\end{itemize}

وجود مكتشف الأخطاء لا يعنى أن تتصرف بتهوّر و تسرع في كتابة برنامج مليء بالأخطاء، بل تريّث و كن هادئاً.

من الآن لدينا خياران :

\begin{itemize}
  \item إمّا أن نحصل على البرامج الثلاثة متفرّقة و هذه هي الطريقة الأكثر تعقيدا، و لكنّها تعمل. على
\textenglish{GNU/Linux}
تحديدا، عدد كبير من المبرمجين يفضّلون استخدام كلّ برنامج على حدة. لن أشرح هذه الطريقة هنا، بل سأتحدّث عن الطريقة الأسهل.
  \item أو أن تحصل على برنامج "ثلاثة في واحد" يتضمّن محرّر النصوص و المترجم و المنقّح. هذا النوع من البرامج يعرف باسم "بيئات التطوير المتكاملة"
(\textenglish{Integrated Development Environments})
و تسمّى اختصارا
\textenglish{IDE}.
\end{itemize}

يوجد العديد من بيئات التطوير. بداية قد تواجه صعوبة في اختيار البيئة الملائمة لك. الشيء الأكيد هو : أي بيئة مهما كانت ستحقق لك العمل المطلوب.

\subsection{اختيار البيئة الخاصة بك}

بدا لي أنه من الأفضل أن أريك بعضا من البيئات الشهيرة و المجانيّة في نفس الوقت. شخصيّا، أنا أستخدمها جميعا و أختار في كل يوم  واحدا منها.

\begin{itemize}
  \item أحد هذه البيئات الّتي أفضّلها هو
\textenglish{Code::Blocks}.
هو مجّاني و يعمل على أغلب أنظمة التشغيل. أنصح كلّ مبتدئ أن يختاره للبدء (و في ما بعد أيضا إذا شعرت أنّه يلائمك جيّدا !).

يعمل على أنظمة التشغيل
\textenglish{Windows}،
\textenglish{Mac OS}
و
\textenglish{GNU/Linux}.
  \item الأكثر شهرة على
\textenglish{Windows}
هو الّذي أنشأته
\textenglish{Microsoft}،
إنّه
\textenglish{Visual C++}.
هو برنامج مدفوع (و باهظ الثمن) لكن لحسن الحظّ توجد نسخة مجانية منه تسمّى
\textenglish{Visual Studio Express}
(أنا أستخدم النسخة القديمة
\textenglish{Visual C++ Express}
في هذا الكتاب). و هي ممتازة جدّا (بينها و بين النسخة المدفوعة فوارق طفيفة). إنه برنامج كامل و يملك منقّحا قويّا.

يعمل على
\textenglish{Windows}
فقط.
  \item على
\textenglish{Mac OS X}
يمكنك استخدام
\textenglish{XCode}
الّذي يفترض أن يكون متوفّرا على قرص تثبيت النظام. يناسب كثيرا مبرمجي
\textenglish{Mac}.

يعمل على
\textenglish{Mac OS X}
فقط.
\end{itemize}

\begin{information}
  ملاحظة لمستخدمي
  \textenglish{GNU/Linux} :
  يوجد العديد من البيئات لهذا النظام، و لكن المبرمجين المحترفين قد يفضّلون تجاوز البيئات و القيام بالترجمة "يدويّا"، و هو شيء أصعب قليلا. نحن سنبدأ باستخدام بيئات التطويرية. لذلك أنصحك بثبيت
  \textenglish{Code::Blocks}
  إن كنت على
  \textenglish{GNU/Linux}
  لكي تتمكن من متابعة شروحاتي.
\end{information}

\begin{question}
من هي البيئة الأفضل من بين كلّ بيئات التطوير هذه؟
\end{question}


كل واحدة من هذه البيئات تمكنك من البرمجة و متابعة بقيّة الدرس من دون أيّة مشاكل. بعضها كامل أكثر من ناحية المميزات، و أخرى سهلة الإستخدام أكثر، ولكن في كلّ الأحوال البرامج الّتي تنشؤها تكون ذاتها أيّا كانت البيئة التي اِخْتَرْتها. فهذا الخيار ليس بالأهمّية الّتي تعتقدها.

في هذا الكتاب سوف أستخدم
\textenglish{Code::Blocks}.
فإن أردت الحصول على نفس لقطات الشاشة خاصّتي، خصوصا لكي لا تضيع في البداية، أنصحك بِدَايَةً بتثبيت
\textenglish{Code::Blocks}.
