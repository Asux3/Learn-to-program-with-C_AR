\chapter{البرمجة المجزّأة
(\textenglish{Modular programming})}
في هذه المرحلة الثانية، سنكتشف مبادئ متقدّمة في لغة الـ
\textenglish{C}
 لن أخفي عليك، هذه المرحلة صعبة الفهم و تحتاج منك التركيز. في نهاية المرحلة، ستكون قادراً على تدبّر أمرك في معظم البرامج المكتوبة بلغة السي. في المرحلة التي تليها نتعلّم كيف نفتح نافذة، كيف ننشئ لعبة ثنائية الأبعاد... الخ

لحدّ الآن عملنا في ملف واحد سمّيناه
\InlineCode{main.c}
. كان أمراً مقبولاً لحدّ الآن لأن برامجنا كانت صغيرة، لكنها ستصبح في القريب العاجل مركّبة من عشرات، لن أقول من مئات الدوال، و إن كنت تريد وضعها كلّها في نفس الملف، فإن هذا الأخير سيصبح ضخماً جداً. لهذا السبب تم اختراع ما نسمّيه بالبرمجة المجزّأة. المبدأ سهل: بدل أن نضع كل الشفرة المصدرية في ملف واحد
\InlineCode{main.c}
، سنقوم بتفريقها إلى عدة ملفات.

\section{النماذج (\textenglish{prototypes})}
لحدّ الآن، كنت عندما تنشئ دالة، أطلب منك وضعها قبل الدالة الرئيسية
\InlineCode{main}
. لماذا؟

لأن للترتيب أهمية حقيقية هنا: فإن قمت بوضع الدالة قبل الـ
\InlineCode{main}
في الشفرة المصدرية، سيقرؤها الجهاز و يتعرف عليها. حينما تقوم باستدعاء الدالة داخل الـ
\InlineCode{main}
، سيعرفها الجهاز و يعرف أيضاً أين يبحث عليها.\\
بالعكس، لو تضع الدالة بعد الـ
\InlineCode{main}
، لن يعمل البرنامج لأن الجهاز لم يتعرّف بعد على الدالة. جرّب ذلك و سترى.
\begin{question}
  لكنه تصميم سيّء نوعاً ما، أليس كذلك ؟
\end{question}
أنا متفق معك! لكن انتبه المبرمجون لهذه النقطة قبلك و عملوا على حلّ المشكل.

بفضل ما سأعلمك إياه الآن، ستتمكن من الدوال في أي ترتيب كان في الشفرة المصدرية، هكذا لن تقلق من هذه الناحية.

\subsection{استعمال النموذج للتصريح عن دالة}
سنقوم بتصريح دوالنا للحاسوب، و هذا بكتابة ما نسميه بـ
\textbf{النماذج}
.لا تنبهر بهذا الاسم، إنه يخبّئ معلومة بسيطة جداً.

تأمل في السطر الأول من دالتنا
\InlineCode{rectangleSurface}
\begin{Csource}
double rectangleSurface(double width, double height)
{
	return width * height;
}
\end{Csource}
قم بنسخ السطر الأول
(\InlineCode{double rectangleSurface...})
المتواجد أعلى الشفرة المصدرية (مباشرة بعد تعليمات التضمين
\InlineCode{\#include}
). أضف
\textbf{فاصلة منقوطة}
في نهاية هذا السطر.\\
و هكذا يمكنك أن تضع الدالة الخاصة بك
\InlineCode{rectangleSurface}
بعد الدالة
\InlineCode{main}
ان أردت !

هذا ما يجب أن تكون عليه الشفرة المصدرية :
\begin{Csource}
#include <stdio.h>
#include <stdlib.h>
// The next line represents the prototype of the function rectangleSurface :
double rectangleSurface(double width, double height);
int main(int argc, char *argv[])
{
	printf("width = 5 and height = 10. Surface = %f\n", rectangleSurface(5, 10));
	printf("width = 2.5 and height = 3.5. Surface = %f\n", rectangleSurface(2.5, 3.5));
	printf("width = 4.2 and height = 9.7. Surface = %f\n", rectangleSurface(4.2, 9.7));

	return 0;
}
// Now, we can put our function wherever we want in the source code:
double rectangleSurface(double width , double height )
{
	return width * height ;
}
\end{Csource}
الشيء الذي تغيّر هنا هو إضافة النموذج أعلى الشفرة المصدرية.\\
النموذج هو عبارة عن إشارة للجهاز، يوحي إليه بوجود دالة تسمى
\InlineCode{rectangleSurface}
و التي تأخذ معاملات إدخال معينة و تُرجِع مخرج من نوع أنت من تحدده.  هذا يساعد الجهاز على تنظيم نفسه.

بفضل ذلك السطر، يمكنك الآن وضع دوالك في أي ترتيب كان دون أي تفكير زائد.

أكتب دائما النموذج الخاص بدوالك. البرامج التي ستكتبها من الآن و صاعداً ستصبح أكثر تعقيداً و تستعمل الكثير من الدوال: من الأحسن أن تتعلّم منذ الآن العادة الجيدة  بوضع نموذج لكل دالة في الشفرة المصدرية.

كما ترى، الدالة
\InlineCode{main}
لا تملك أي نموذج، و كمعلومة فهي الوحيدة التي لا تملك نموذجاً ! لأن الجهاز يعرفها (فهي نفسها مكررة في جميع البرامج).

عليك أن تعرف أنه في سطر النموذج، لست مضطراً إلى تحديد المعاملات التي تتلقاها الدالة كمدخل. الجهاز يحتاج أن يتعرّف إلى نوع المداخل فقط.

يمكننا أن نكتب ببساطة :
\begin{Csource}
double rectangleSurface (double, double);
\end{Csource}
و مع ذلك، فالطريقة التي أريتك إياها أعلاه تعمل أيضاً. الشيء الجيد فيها هو أن كلّ ما عليك فعله هو نسخ و لصق السطر الأول الخاص بالدالة مع إضافة فاصلة منقوطة (طريقة سهلة و سريعة).
\begin{critical}
  لا تنس
\underline{أبدا}
وضع فاصلة منقوطة بعد النموذج، هذا يمكّن الحاسوب من التفريق بين النموذج و بداية الدالة.\\
إن لم تفعل، ستعترضك أخطاء غير مفهومة أثناء عملية الترجمة.
\end{critical}
