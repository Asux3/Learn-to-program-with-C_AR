\chapter{البرمجة المجزّأة
(\textenglish{Modular programming})}
في هذه المرحلة الثانية، سنكتشف مبادئ متقدّمة في لغة الـ
\textenglish{C}
 لن أخفي عليك، هذه المرحلة صعبة الفهم و تحتاج منك التركيز. في نهاية المرحلة، ستكون قادراً على تدبّر أمرك في معظم البرامج المكتوبة بلغة السي. في المرحلة التي تليها نتعلّم كيف نفتح نافذة، كيف ننشئ لعبة ثنائية الأبعاد... الخ

لحدّ الآن عملنا في ملف واحد سمّيناه
\InlineCode{main.c}
. كان أمراً مقبولاً لحدّ الآن لأن برامجنا كانت صغيرة، لكنها ستصبح في القريب العاجل مركّبة من عشرات، لن أقول من مئات الدوال، و إن كنت تريد وضعها كلّها في نفس الملف، فإن هذا الأخير سيصبح ضخماً جداً. لهذا السبب تم اختراع ما نسمّيه بالبرمجة المجزّأة. المبدأ سهل: بدل أن نضع كل الشفرة المصدرية في ملف واحد
\InlineCode{main.c}
، سنقوم بتفريقها إلى عدة ملفات.
