\chapter{المكدّسات و الطوابير (\textenglish{Stacks and Queues})}

لقد اكتشفنا مع القوائم المتسلسلة طريقة جديدة أكثر مرونة من الجداول لتخزين البيانات. هذه القوائم مرنة بشكل خاص لأنه يمكننا أن نُدرج فيها و نحذف منها بيانات من أي مكان أردنا و في أية لحظة.

المكدّسات و الطوابير التي سنكتشفها هنا هما شكلان خاصّان نوعاً ما من القوائم المتسلسلة. فهما تسمحان بالتحكّم بالطريقة التي تُضاف بها العناصر الجديدة إليها. هذه المرة لن نقوم بإضافة عناصر جديدة في وسط القائمة، بل فقط في البداية أو النهاية. 

المكدّسات و الطوابير تعتبران مفيدتان للغاية من أجل البرامج التي تحلل المعطيات الّتي تصل بالتدريج. سنرى بالتفصيل كيف تعملان في هذا الفصل.

تتشابه المكدّسات و الطوابير كثيراً، لكنهما تختلفان اختلافاً بسيطاً ستتعرف عليه بسرعة. سنكتشف أولاً المكدّسات و التي ستذكّرك بالقوائم المتّصلة بشكل كبير. 

بشكل عام، سيكون هذا الفصل بسيطاً إذا كنت قد فهمت جيّداً كيفية عمل القوائم المتسلسلة. إن لم تكن هذه حالتك، فأعد قراءة الفصل السابق لأننا سنحتاج إليه.
