\chapter{معالجة الأحداث}

معالجة الأحداث هو من أهم الأساسيات في الـ\textenglish{SDL}.\\
و ربّما قد يكون الشطر الأكثر شغفاً لاكتشافه. لأنه انطلاقا من هنا ستبدأ فعلاً في التحكّم في تطبيقك.

كلّ من مرفقات الحاسوب (فأرة، لوحة مفاتيح، \dots) قادرة على إنتاج حدث. سنتعلّم كيف نستقبل كل حدث و نتعامل معه. تطبيقك سيصبح أخيراً تفاعليّا !

فعلياً، ما هو الحدث ؟ الحدث هو عبارة عن إشارة
(\textenglish{signal})
يتم إرسالها عن طريق إحدى مرفقات الحاسوب 
(\textenglish{peripherals})
(أو عن طريق نظام التشغيل بذاته) إلى التطبيق. هذه أمثلة عن بعض الأحداث المألوفة :

\begin{itemize}
	\item حينما يضغط المُستعمل على زر من لوحة المفاتيح.
	\item و أيضاً حينما ينقر بالفأرة.
	\item حينما يحرّك الفأرة.
	\item حينما يقوم بتصغير النافذة.
	\item حينما يطلب إغلاق النافذة.
	\item إلى آخره.
\end{itemize}

الهدف من هذا الدرس هو تعلّم كيفية معالجة الأحداث. يمكنك أخيراً القول للحاسوب : "إذا نقر المستعمل في هذا المكان، قم بفعل كذا، و إن لم يفعل، قم بكذا. إذا حرّك الفأرة، قم بكذا. إذا ضغط على الزر
\InlineCode{Q}،
أوقف البرنامج. إلخ".
