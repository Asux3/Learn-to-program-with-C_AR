\chapter{عالم المتغيّرات}

تعلّمت كيفية إظهار نصّ على الشاشة. جيد، لكنّ هذا ليس شيئا مهماً. هذا لأنك لا تعرف بعد ما يدعى بـ
\underline{المتغيّرات}
(\textenglish{Variables})
في البرمجة.

فائدة هذه المتغيرات هي تمكين الحاسوب من حفظ أعداد في الذاكرة. سنبدأ ببعض الشرح حول ذاكرة الحاسوب وكيفيّة عملها. قد يبدو هذا بسيطا جدّا للبعض، لكنّي أفترض أنّك لا تعرف شيئا عن ذاكرة الحاسوب.

\section{أمر متعلق بالذاكرة}
ما سأعلمك في هذا الدرس هو أمر له علاقة مباشرة بذاكرة حاسوبك.

كل إنسان حيّ له ذاكرة. الأمر عينه بالنسبة للحاسوب، لكن الحاسوب له أنواع عديدة من الذاكرة.

\begin{question}
  لم يملك الحاسوب أنواع عديدة من الذاكرة، واحدة يمكنها أن تكفي، أليس الأمر كذلك؟
\end{question}
كلّا: المشكلة أننا نحتاج ذاكرة سريعة (لاسترجاع المعلومات بسرعة) وفي نفس الوقت كبيرة (لحفظ بيانات كثيرة) قد تضحك إن أخبرتك أننا حتى اليوم لم نتمكن من صنع ذاكرة بهذه المواصفات. أو بالأحرى الذاكرة السريعة باهظة الثمن لذلك لا يتم إنتاج الكثير منها.

لذلك نجد في الحواسيب الحديثة ذاكرة سريعة جدا لكنها ليس ذات سعة كبيرة، وأخرى ذات سعة كبيرة جدّا لكنها غير سريعة.

\subsection{الأنواع المختلفة من الذاكرة}
كي أوضح لك الصورة أكثر، إليك أنواع الذاكرة الموجودة في الحاسوب، من الأسرع إلى الأبطأ:
\begin{enumerate}
  \item السجلات (
\textenglish{Registers}
): ذاكرة سريعة جدّا، موجودة داخل المعالج.
  \item ذاكرة التخبئة (
\textenglish{Cache memory}
): تمثل همزة وصل بين السجلات والذاكرة الحية.
  \item ذاكرة الوصول العشوائي (
\textenglish{Random access memory}
): وهي الذاكرة التي نستخدمها كثيرا، وتدعى اختصارا
\textenglish{RAM}.
  \item القرص الصلب (
\textenglish{Hard disk}
): والذي تعرفه بالطبع، نستعمله لحفظ الملفات.
\end{enumerate}
كما قلت لك، لقد رتبتها من الأسرع (السجلات) إلى الأبطأ (القرص الصلب)، وإن كنت قد تابعت جيدا فقد فهمت أن الذاكرة الأصغر هي الأسرع والأبطأ هي الأكبر.\\
السجلات لا تسع إلا لحمل بضعة أعداد أما القرص الصلب فيمكنه تخزين ملفات ضخمة.

\begin{information}
   عندما أقول ذاكرة بطيئة فهذا بالنسبة لحاسوبك، ففي نظر الحاسوب استغراق 8 ميلي ثانية للوصول إلى القرص الصلب يعتبر زمنا طويلا جدّا!
\end{information}

ما الذي يجب أن أتذكره من كل هذا؟\\
أردت أن أخبرك أننا في الدروس القادمة سوف نستخدم ذاكرة الوصول العشوائي كثيرا. سنتعلم أيضا كيفية القراءة والكتابة في الملفات على القرص الصلب (ليس الآن، لا يزال الوقت مبكّرا على هذا). أمّا بخصوص السجلّات وذاكرة التخبئة فلن نتعامل معهما مطلقا، فالحاسوب هو من سيهتم بأمرهما.

\begin{information}
  في لغات البرمجة منخفضة المستوى، كلغة التجميع (
\textenglish{Assembly language}
) نتعامل مباشرة مع السجلّات، لقد درستها، ويمكنني أن أقول لك أن القيام بعملية ضرب بسيطة يتطلب مجهودا! لحسن الحظ ففي لغة
\textenglish{C}
 (وفي أغلب اللغات الأخرى) الأمر أسهل من ذلك بكثير.
\end{information}

يجب إضافة شيء مهمّ آخر: القرص الصلب هو الوحيد الذي يمكنه حفظ المعلومات بشكل دائم.
\textbf{كل أنواع الذاكرات الأخرى مؤقتة، فبمجرد إطفاء الحاسوب تفقد كل محتواها}!

لحسن الحظ فعند إعادة تشغيل الحاسوب يقوم القرص الصلب بتذكيرها بمحتواها.

\subsection{صورة لذاكرة الوصول العشوائي}
نظرا لأننا سنستعمل ذاكرة الوصول العشوائي خلال لحظات، فمن الأفضل أن أريها لكم (مؤطر بالأحمر):
\Picture{Chapter_I-4_Computer}
لا أطلب منك معرفة كيفية عملها، لكن أردت فقط أن أريك مكانها داخل جهازك. وهذه صورة مقربة لإحدى أشرطتها:
\Picture{Chapter_I-4_RAM}
وهي تدعى اختصارا
\textbf{\textenglish{RAM}}
، لذلك لا تحتر إن سميتها هكذا لاحقا. بالنسبة للذاكرات الأخرى (السجلات والتخبئة) فهي صغيرة لدرجة أنه لا يمكن رؤيتها بالعين المجرّدة.

\subsection{مخطط ذاكرة الوصول العشوائي}
عرض المزيد من الصور لن يفيدك كثيرا، لكن يجب عليك فهم كيف تعمل من الداخل، لذلك سأقدم لك هذا المخطط البسيط الذي يمثل هندسة ذاكرة الوصول العشوائي:
\Picture{Chapter_I-4_RAM-Schema}

كما ترى، يمكننا أن نميز عمودين:
\begin{itemize}
  \item هناك
\textbf{العناوين}
: هي أعداد تسمح للحاسوب بتحديد موضع القيم في الـ
\textenglish{RAM}
. نبدأ بالعنوان 0 وننتهي بالعنوان 3,448,765,900,126 وبعض الأجزاء. لا أعلم بالضبط كم عدد العناوين الموجودة في الـ
\textenglish{RAM}
، لكني أعرف أنها كثيرة جدا. إضافة إلى ذلك، هذا أمر يتعلق بكمية الذاكرة الموجودة في جهازك، فكلما زادت الذاكرة زادت معها العناوين وصار بإمكاننا تخزين معلومات أكثر.
  \item عند كل عنوان يمكننا تخزين
\textbf{قيمة}
(عدد). حاسوبك يقوم بتخزين هذه الأعداد في ذاكرة الوصول العشوائي لكي يتمكن من تذكرها. ولا يمكننا تخزين سوى عدد واحد عند كل عنوان.
\end{itemize}

لا يمكن للذاكرة الحية تخزين شيء سوى الأعداد.

\begin{question}
  لكن كيف يمكننا تخزين الكلمات؟
\end{question}

سؤال جيد. في الواقع حتى الحروف ليست سوى أعداد في نظر الحاسوب! الجملة هي مجرد تتابع لأعداد.\\
يوجد جدول يوافق بين الأعداد والحروف، جدول يقول مثلا بأن العدد 67 يوافق الحرف
\textenglish{Y}
. لن أدخل في التفاصيل أكثر، ستكون لنا فرصة للرجوع إلى هذا لاحقا.

فلنعد إلى مخططنا، الأمور بسيطة جدا: إذا أراد الحاسوب تذكر العدد 5 (الذي قد يمثل عدد الأرواح المتبقية لشخصية في لعبة) فسوف يضعه في مكان ما في الذاكرة أين يتوفر مكان شاغر ويحفظ العنوان الموافق (مثلا 3,062,199,902). لاحقا، عندما يريد معرفة هذا العدد فسيذهب إلى خانة الذاكرة التي تحمل العنوان رقم 3,062,199,902 وسيجد القيمة 5.

هذه آلية عمل الذاكرة بشكل عام. قد يكون الأمر لا زال غامضا في ذهنك حاليا (ما فائدة تخزين عدد إن كان علينا تذكر عنوانه بدلا من ذلك؟) لكن كل شيء سيتضح مع بقية الدروس، أنا أعدك!

\section{التصريح عن متغير}
صدّقني هذه المقدّمة القصيرة عن الذاكرة ستكون مهمّة أكثر مما تعتقد. الآن يمكننا العودة إلى البرمجة.

إذن، ما هو
\undeline{المتغير}
(\textenglish{Variable}) ؟\\
إنه معلومة صغيرة نخزنها مؤقتا في الذاكرة الحية. ببساطة يمكننا القول إن المتغير هو قيمة يمكن أن تتغير أثناء اشتغال البرنامج. مثلا عددنا 5 الذي ذكرناه سابقا يمكن أن يتناقص بمرور الزمن. إذا وصل إلى العدد 0 فسنعرف أن اللاعب قد خسر.

في برامجنا سيكون هناك الكثير من المتغيرات. ستراها في كلّ مكان.

في لغة السي، المتغير يتميز بشيئين:
\begin{itemize}
  \item \underline{قيمة}
: هو العدد الذي يحويه، 5 مثلا.
  \item \underline{اسم}
: وهو الذي يمكننا من معرفة المتغيّر. في البرمجة لن يكون علينا تذكّر عناوين الذاكرة. بدلا من ذلك علينا فقط استخدام أسماء المتغيرات. المترجم هو من سيقوم بتحويل الأسماء إلى عناوين.
\end{itemize}

\subsection{إعطاء اسم للمتغير}
في لغة البرمجة
\textenglish{C}
كل متغير يجب أن يملك اسما خاصا به. ومن أجل متغيرنا الذي يحوي عدد الأرواح المتبقية للاعب يمكننا أن نسميه
"\textenglish{Number of lives}"
أو شيء من هذا القبيل.

للأسف توجد بعض الشروط، لا يمكنك تسمية المتغير كيفما شئت:
\begin{itemize}
  \item لا يجب أن يحتوي الاسم سوى على الحروف الصغيرة والكبيرة والأرقام
(\InlineCode{abcABC012}).
  \item يجب أن يبدأ الاسم بحرف.
  \item المسافات ممنوعة. بدلا من ذلك يمكننا استخدام الحرف المعروف باسم
\textenglish{underscore}
 (\InlineCode{\_}).
إنه الحرف الخاص الوحيد غير الحروف والأرقام الذي يمكن استعماله في اسم متغير.
  \item لا يمكنك استخدام حروف غير الحروف الإنجليزية.
\end{itemize}

وأخيرا يجب أن تعرف أن لغة
\textenglish{C}
 تفرّق بين الحروف الصغيرة والكبيرة. ولثقافتك، نقول إن
\textenglish{C}
 حساسة لحالة الأحرف
(\textenglish{Case sensitive}).
كمثال، الأسماء
\InlineCode{width}
 أو
\InlineCode{WIDTH}
 أو
\InlineCode{WiDth}
تعتبر أسماء متغيرات مختلفة، حتى لو كانت تعني لنا الأمر نفسه.

هذه أمثلة عن أسماء متغيرات صالحة:
\InlineCode{numberOfLives}،
\InlineCode{name}،
\InlineCode{surname}،
\InlineCode{phone\_number}،
\InlineCode{phoneNumber}.

لكل مبرمج طريقة خاصة في كتابة أسماء المتغيرات. خلال هذا الدرس سأريك طريقتي:
\begin{itemize}
  \item أبدأ دائما بحرف صغير.
  \item إن كان في الاسم أكثر من كلمة أضع حرف كبيرا في بداية كلّ كلمة.
\end{itemize}

أطلب منك كتابة أسماء متغيراتك بنفس الطريقة التي أتبعها، هذا لكي نكون على تفاهم.

\begin{critical}
  أيّا كان اختيارك، فعليك دائما إعطاء أسماء واضحة لمتغيراتك. كان بإمكاننا اختصار
\InlineCode{numberOfLives}
إلى
\InlineCode{nol}
مثلا. هذا أقصر في الكتابة، لكنه أقل وضوحا عندما تعيد قراءة الشفرة المصدرية. فأنصحك بإعطاء أسماء أطول لمتغيراتك إن كان ذلك يحسّن فهمها.
\end{critical}
