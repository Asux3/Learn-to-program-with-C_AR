\chapter*{مقدّمة}
إن التحرّر الفكري في بداية القرن العشرين أدّى إلى توسّع في البحوث العلمية التي شملت كل الميادين لاسيّما التكنولوجية منها كعلوم الحاسوب. هذه الأخيرة أعقبتها ثورة في لغات البرمجة التي تعتبر ركيزة أساسية تقوم عليها البرامج. من بين هذه اللغات نجد لغة الـ
\textenglish{C}
، إذ تعتبر من أقوى لغات البرمجة و أكثرها شيوعاً، فهي مستلهمة من طرف لغتي
 \textenglish{B}
 و
 \textenglish{BCPL}
حيث تمّ تطويرها في عام 1972 من طرف
\textenglish{Ken Thompson}
 و
 \textenglish{Dennis Ritchie}
، و في ظرف سنة واحدة توسّعت لتكون عِـماد نظام التشغيل
\textenglish{UNIX}
 بنسبة
 90\%
ثم تم توزيعها في العام المـُوالي رسمياً عبر الجامعات لتصبح بذلك لغة برمجة عالمية. و اشتهرت لغة الـ
\textenglish{C}
 كونـُها لغة عالية المستوى، لها مُترجم سريع و فعّال. كما أنها لغة برمجية نقّالة، هذا يعني أن أي برنامج يحترم المعيار
AINSI
 يمكن أن يتمّ تشغيله على أيّة منصّة تحتوي على مترجم
\textenglish{C}
 دون أيّة تخصيصات.

يعتبر هذا الكتاب بوابة سهلة لكلّ مبتدئ لتعلّم لغة الـ
\textenglish{C}
 خطوة بخطوة بدءاً من الأساسيات وصولاً إلى تطوير ألعاب ثنائية الأبعاد و التحكّم في هياكل البيانات الأكثر تعقيداً. الكتاب مرفق بجملة من التمارين و الأعمال التطبيقية المحلولة التي تساعد على هضم المفاهيم المكتسبة و تطبيقها على أيّ مشكل برمجي مهما كان نوعه. و لأن الكثير من لغات البرمجة تعتمد أساساً على الـ
\textenglish{C}
كالـ
\textenglish{Java}
و الـ
\textenglish{C++}
 و الـ
\textenglish{C\#}
 ( لغات برمجية غرضية التوجّه ) و حتى
 \textenglish{PHP}
 ( لغة لبرمجة المواقع ) فإن تعلّم لغة الـ
\textenglish{C}
 سيساعد على تعلّم أيّة لغة برمجية مهما كانت. تبقى الإرادة و حبّ العمل و الشغف المفاتيح الرئيسية للنجاح و الوصول إلى الاحترافية.

\hfill المُـترجمة.
