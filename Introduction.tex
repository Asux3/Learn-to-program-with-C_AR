\chapter*{تقديم}
إن التحرّر الفكري في بداية القرن العشرين أدّى إلى توسّع في البحوث العلمية التي شملت كل الميادين لا سيّما التكنولوجية منها كعلوم الحاسوب. هذه الأخيرة أعقبتها ثورة في لغات البرمجة التي تعتبر ركيزة أساسية تقوم عليها البرامج. من بين هذه اللغات نجد لغة \textenglish{C}،
إذ تعتبر من أقوى لغات البرمجة وأكثرها شيوعًا، فهي مستلهمة من طرف لغتي
 \textenglish{B}
 و
 \textenglish{BCPL}
حيث تمّ تطويرها في عام 1972 من طرف
\textenglish{Ken Thompson}
و
 \textenglish{Dennis Ritchie}،
و في ظرف سنة واحدة توسّعت لتكون عِـماد نظام التشغيل
\textenglish{UNIX}
بنسبة
90\%
ثم تم توزيعها في العام المـُوالي رسميًا عبر الجامعات لتصبح بذلك لغة برمجة عالمية. واشتهرت لغة \textenglish{C}
 كونـُها لغة عالية المستوى، لها مُترجم سريع و فعّال. كما أنها لغة برمجية نقّالة، هذا يعني أن أي برنامج يحترم المعيار
\textenglish{AINSI}
يمكن أن يتمّ تشغيله على أيّة منصّة تحتوي على مترجم
\textenglish{C}
 دون أيّة تخصيصات.

يعتبر هذا الكتاب بوابة سهلة لكلّ مبتدئ لتعلّم لغة \textenglish{C}
خطوة بخطوة بدءً من الأساسيات وصولًا إلى تطوير ألعاب ثنائية الأبعاد والتحكّم في هياكل البيانات الأكثر تعقيدًا. الكتاب مرفق بجملة من التمارين والأعمال التطبيقية المحلولة التي تساعد على هضم المفاهيم المكتسبة وتطبيقها على أيّ مشكل برمجي مهما كان نوعه. ولأن الكثير من لغات البرمجة تعتمد أساسًا على \textenglish{C}
مثل \textenglish{Java}
و \textenglish{C++}
و \textenglish{C\#}
(لغات برمجية غرضية التوجّه) وحتى
\textenglish{PHP}
(لغة لبرمجة المواقع) فإن تعلّم لغة \textenglish{C}
 سيساعد على تعلّم أيّة لغة برمجية كانت. تبقى الإرادة وحبّ العمل والشغف المفاتيح الرئيسية للنجاح والوصول إلى الاحترافية.

\vfill

\hfill\parbox{0.3\textwidth}{\centering
عدن بلواضح

\vspace{1em}
الجزائر\\[0.5em]
في
24 ذو القعدة 1438\\[0.3em]
الموافق لـ17 أوت 2017
%\Hijritoday\\[0.3em]
%الموافق لـ\today

}

