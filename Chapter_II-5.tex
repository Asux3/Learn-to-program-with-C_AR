\chapter{المعالج القبلي}
بعد كل المعلومات المتعبة التي تلقّيتها في الدروس حول الجداول، النصوص و المؤشرات، فسنقوم بالتوقف قليلا. لقد تعلّمت أشياء جديدة كثيرة في الفصول السابقة، لن يكون لديّ مانع من نسترجع أنفاسنا قليلا.

هذا الفصل يتحدّث عن المعالج القبلي، هذا البرنامج الّذي يعمل مباشرة قبل الترجمة.\\
لا تخطئ : المعلومات التي به ستكون مهمّة لك. لكنّها ستكون أقل تعقيداً من الّتي تعلّمتها مؤخّراً.

\section{الـ\texttt{include}}
كما شرحت لكم في الفصول الأولى من الكتاب، نجد في الشفرات المصدريّة سطورا خاصّة تسمّى بـ\textbf{توجيهات المعالج القبلي} (\textenglish{Preprocessor directives}).\\
هذه السطور لديها الخاصيّة التالية : تبدأ دائما بالرمز
\InlineCode{\#}.
لذا فمن السهل التعرّف عليها.

التوجيهة الوحيدة التي رأيناها لحدّ الآن هي
\InlineCode{\#include}.\\
هذه التوجيهة تسمح لنا بتضمين محتوى ملف في آخر. قلت لكم هذا من قبل.\\
نحن نحتاجها في تضمين الملفات ذات الصيغة
\InlineCode{.h}
كملفات
\InlineCode{.h}
الخاصّة بالمكتبات
(\InlineCode{stdlib.h}، \InlineCode{stdio.h}، \InlineCode{string.h}، \InlineCode{math.h}...)،
و أيضاً ملفات
\InlineCode{.h}
الخاصّة بنا.

لنضمّن ملفاً ذو صيغة
\InlineCode{.h}
موجوداً في نفس المجلّد الذي ثبتنا فيه الـ\textenglish{IDE}
(أي البيئة التطويرية كالـ\textenglish{Code::Blocks}
مثلا)، نستعمل علامات الترتيب
\InlineCode{< >}
كالتالي :
\begin{Csource}
#include <stdlib.h>
\end{Csource}
بينما لتضمين ملفّ
\InlineCode{.h}
موجود في المجلّد الذي به مشروعنا، فسنقوم يذلك باستخدام علامتي الترتيب كالتالي :
\begin{Csource}
#include "myfile.h"
\end{Csource}
في الحقيقة، المعالج القبلي يتمّ تشغيله قبل الترجمة. يبحث في كلّ ملفاتك عن توجيهات المعالج القبلي، تلك الأسطر المشهورة التي تبدأ بـ\InlineCode{\#}.\\
عندما يجد التوجيهة
\InlineCode{\#include}،
يقوم بإدراج محتوى الملفّ في مكان وجود
\InlineCode{\#include}.


افترض أن لديّ ملفّا
\InlineCode{file.c}
يحتوي الشفرة الخاصة بالدوال التي كتبتها، و لدي ملف
\InlineCode{file.h}
يحتوي نماذج الدوال التي هي موجودة بالملف
\InlineCode{file.c}،
يمكن تلخيص ذلك بالمخطط التالي.
\Picture{Chapter_II-5_file-c-file-h}
كل محتوى الملف
\InlineCode{file.h}
سيتم وضعه داخل الملف
\InlineCode{file.c}
في مكان التوجيهة
\InlineCode{\#include "file.h"}.

تخيّل أن لدينا في الملف
\InlineCode{file.c}
التالي :
\begin{Csource}
#include "file.h"

int myFunction(int something, double stupid)
{
  /* The code of the function */
}
void anotherFunction(int value)
{
  /* The code of the function */
}
\end{Csource}
و في الملف
\InlineCode{file.h} :
\begin{Csource}
int myFunction(int something, double stupid);
void anotherFunction(int value);
\end{Csource}
عندما يمر المعالج القبلي بهذه الشفرة، قبل أن تتم ترجمة الملف
\InlineCode{file.c}،
سيضع كما قلت محتوى الملف
\InlineCode{file.h}
 في الملف
\InlineCode{file.c}.
في النهاية، يعني أن الملف
\InlineCode{file.c}
\textit{قُبَيْل}
الترجمة سيحتوي التالي :
\begin{Csource}
int myFunction(int something, double stupid);
void anotherFunction(int value);

int myFunction(int something, double stupid)
{
  /* The code of the function */
}
void anotherFunction(int value)
{
  /* The code of the function */
}
\end{Csource}
محتوى
\InlineCode{.h}
تمّ ادخاله مكان
\InlineCode{\#include}.

هذا ليس بالأمر المعقد لفهمه، و لعلّ بعض القراء يشكك في أن الأمر يحصل بهذه الطريقة.\\
مع هذه الشروحات الإضافيّة، أتمنّى أنّ الجميع يوافقني.
الـ\InlineCode{\#include}
لا تفعل أي شيء سوى إحضار محتوى ملف و تضمينه في آخر، من المهمّ فهم هذا الأمر جيّدا.
\begin{information}
  إن كنّا قد قررنا وضع النماذج في ملفّات
\InlineCode{.h}
بدل ملفّات
\InlineCode{.c}،
فهذا من المبدأ.
بالطبع، كان بإمكاننا وضع نماذج الدوال في أعلى الملفات
\InlineCode{.c}
بأنفسنا (قد نفعل هذا أحيانا في بعض البرامج الصغيرة)، لكن لأسباب تنظيميّة، من المنصوح به جدّا وضع النماذج في ملفّات
\InlineCode{.h}.
 عندما يكبر برنامجك و يصبح لديك الكثير من ملفّات
\InlineCode{.c}
يعتمدون على نفس
\InlineCode{.h}،
ستكون سعيدا لأنّك لن تظطرّ إلى نسخ و لصق النماذج الخاصّة بنفس الدوال عدّة مرّات !
\end{information}
