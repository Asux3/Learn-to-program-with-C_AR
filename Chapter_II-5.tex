\chapter{المعالج القبلي}
بعد كل المعلومات المتعبة التي تلقّيتها في الدروس حول الجداول، النصوص و المؤشرات، فسنقوم بالتوقف قليلا. لقد تعلّمت أشياء جديدة كثيرة في الفصول السابقة، لن يكون لديّ مانع من نسترجع أنفاسنا قليلا.

هذا الفصل يتحدّث عن المعالج القبلي، هذا البرنامج الّذي يعمل مباشرة قبل الترجمة.\\
لا تخطئ : المعلومات التي به ستكون مهمّة لك. لكنّها ستكون أقل تعقيداً من الّتي تعلّمتها مؤخّراً.

\section{الـ\texttt{include}}
كما شرحت لكم في الفصول الأولى من الكتاب، نجد في الشفرات المصدريّة سطورا خاصّة تسمّى بـ\textbf{توجيهات المعالج القبلي} (\textenglish{Preprocessor directives}).\\
هذه السطور لديها الخاصيّة التالية : تبدأ دائما بالرمز
\InlineCode{\#}.
لذا فمن السهل التعرّف عليها.

التوجيهة الوحيدة التي رأيناها لحدّ الآن هي
\InlineCode{\#include}.\\
هذه التوجيهة تسمح لنا بتضمين محتوى ملف في آخر. قلت لكم هذا من قبل.\\
نحن نحتاجها في تضمين الملفات ذات الصيغة
\InlineCode{.h}
كملفات
\InlineCode{.h}
الخاصّة بالمكتبات
(\InlineCode{stdlib.h}، \InlineCode{stdio.h}، \InlineCode{string.h}، \InlineCode{math.h}...)،
و أيضاً ملفات
\InlineCode{.h}
الخاصّة بنا.

لنضمّن ملفاً ذو صيغة
\InlineCode{.h}
موجوداً في نفس المجلّد الذي ثبتنا فيه الـ\textenglish{IDE}
(أي البيئة التطويرية كالـ\textenglish{Code::Blocks}
مثلا)، نستعمل علامات الترتيب
\InlineCode{< >}
كالتالي :
\begin{Csource}
#include <stdlib.h>
\end{Csource}
بينما لتضمين ملفّ
\InlineCode{.h}
موجود في المجلّد الذي به مشروعنا، فسنقوم يذلك باستخدام علامتي الترتيب كالتالي :
\begin{Csource}
#include "myfile.h"
\end{Csource}
في الحقيقة، المعالج القبلي يتمّ تشغيله قبل الترجمة. يبحث في كلّ ملفاتك عن توجيهات المعالج القبلي، تلك الأسطر المشهورة التي تبدأ بـ\InlineCode{\#}.\\
عندما يجد التوجيهة
\InlineCode{\#include}،
يقوم بإدراج محتوى الملفّ في مكان وجود
\InlineCode{\#include}.


افترض أن لديّ ملفّا
\InlineCode{file.c}
يحتوي الشفرة الخاصة بالدوال التي كتبتها، و لدي ملف
\InlineCode{file.h}
يحتوي نماذج الدوال التي هي موجودة بالملف
\InlineCode{file.c}،
يمكن تلخيص ذلك بالمخطط التالي.
\Picture{Chapter_II-5_file-c-file-h}
كل محتوى الملف
\InlineCode{file.h}
سيتم وضعه داخل الملف
\InlineCode{file.c}
في مكان التوجيهة
\InlineCode{\#include "file.h"}.

تخيّل أن لدينا في الملف
\InlineCode{file.c}
التالي :
\begin{Csource}
#include "file.h"

int myFunction(int something, double stupid)
{
  /* The code of the function */
}
void anotherFunction(int value)
{
  /* The code of the function */
}
\end{Csource}
و في الملف
\InlineCode{file.h} :
\begin{Csource}
int myFunction(int something, double stupid);
void anotherFunction(int value);
\end{Csource}
عندما يمر المعالج القبلي بهذه الشفرة، قبل أن تتم ترجمة الملف
\InlineCode{file.c}،
سيضع كما قلت محتوى الملف
\InlineCode{file.h}
 في الملف
\InlineCode{file.c}.
في النهاية، يعني أن الملف
\InlineCode{file.c}
\textit{قُبَيْل}
الترجمة سيحتوي التالي :
\begin{Csource}
int myFunction(int something, double stupid);
void anotherFunction(int value);

int myFunction(int something, double stupid)
{
  /* The code of the function */
}
void anotherFunction(int value)
{
  /* The code of the function */
}
\end{Csource}
محتوى
\InlineCode{.h}
تمّ ادخاله مكان
\InlineCode{\#include}.

هذا ليس بالأمر المعقد لفهمه، و لعلّ بعض القراء يشكك في أن الأمر يحصل بهذه الطريقة.\\
مع هذه الشروحات الإضافيّة، أتمنّى أنّ الجميع يوافقني.
الـ\InlineCode{\#include}
لا تفعل أي شيء سوى إحضار محتوى ملف و تضمينه في آخر، من المهمّ فهم هذا الأمر جيّدا.
\begin{information}
  إن كنّا قد قررنا وضع النماذج في ملفّات
\InlineCode{.h}
بدل ملفّات
\InlineCode{.c}،
فهذا من المبدأ.
بالطبع، كان بإمكاننا وضع نماذج الدوال في أعلى الملفات
\InlineCode{.c}
بأنفسنا (قد نفعل هذا أحيانا في بعض البرامج الصغيرة)، لكن لأسباب تنظيميّة، من المنصوح به جدّا وضع النماذج في ملفّات
\InlineCode{.h}.
 عندما يكبر برنامجك و يصبح لديك الكثير من ملفّات
\InlineCode{.c}
يعتمدون على نفس
\InlineCode{.h}،
ستكون سعيدا لأنّك لن تظطرّ إلى نسخ و لصق النماذج الخاصّة بنفس الدوال عدّة مرّات !
\end{information}

\section{الـ\texttt{define}}
سنتعرف الآن على توجيهة معالج جديدة و هي
\InlineCode{\#define}.

هذه التوجيهة تسمح بالتصريح عن
\textbf{ثابت معالج قبلي}.
هذا يسمح بإرفاق قيمة بعبارة.\\
إليكم مثالا :
\begin{Csource}
#define INITIAL_NUMBER_OF_LIVES 3
\end{Csource}
يجب أن تكتب بالترتيب :
\begin{itemize}
  \item الـ\InlineCode{define}.
  \item الكلمة التي تريد ربط القيمة بها.
  \item قيمة الكلمة.
\end{itemize}
احذر : رغم التشابه (خصوصا في الاسم الذي اعتدنا كتابته بحروف كبيرة)، فهذه مختلفة كثيرا عن الثوابت التي تعلّمناها حتّى الآن، مثل :
\begin{Csource}
const int INITIAL_NUMBER_OF_LIVES = 3;
\end{Csource}
الثوابت تأخذ حيّزا في الذاكرة. حتى و إن لم تتغير قيمتها فإن العدد 3 مخزّن في مكان ما من الذاكرة. هذا ليس هو الحال مع ثوابت المعالج القبلي !

كيف تعمل ؟ في الواقع، الـ\InlineCode{\#define}
تستبدل في شفرتك المصدريّة كلّ الكلمات بقيمتهم الموافقة. هذا تقريبا مثل عمليّة البحث و الاستبدال
(\textenglish{Search/Replace})
الموجودة في برنامج
\textenglish{Word}
مثلا. إذن السطر :
\begin{Csource}
#define INITIAL_NUMBER_OF_LIVES 3
\end{Csource}
يستبدل في الملف كلّ
\InlineCode{INITIAL\_NUMBER\_OF\_LIVES}
بالعدد 3.

هذا مثال على ملف
\InlineCode{.c}
قبل مرور المعالج القبلي :
\begin{Csource}
#define INITIAL_NUMBER_OF_LIVES 3
int main(int argc, char *argv[])
{
	int lives = INITIAL_NUMBER_OF_LIVES;
  /* Code ... */
\end{Csource}
و بعدما يمرّ المعالج القبلي :
\begin{Csource}
int main(int argc, char *argv[])
{
	int lives = 3;
  /* Code ... */
\end{Csource}
قبل الترجمة، كلّ
\InlineCode{\#define}
يتمّ استبدالها بالقيمة الموافقة. المترجم "يرى" الملفّ بعد مرور المعالج القبلي، حيث تكون الاستبدالات قد تمت.
\begin{question}
ما الفائدة بالنسبة للثوابت التي رأيناها حتّى الآن ؟
\end{question}
كما قلت لك، هي لا تأخذ مكانا في الذاكرة. هذا منطقيّ، نظرا لأنّه عند الترجمة لا يتبقّى سوى الأعداد في الشفرة المصدريّة.

توجد فائدة أخرى و هي أنّ الاستبدال يتمّ في كامل الملف حيث توجد
\InlineCode{\#define}.
إن قمت بتعريف ثابت في الذاكرة داخل دالّة، فلن يكون صالحا إلّا داخل تلك الدالّة، ثمّ يتمّ حذفه بعد نهايتها.
بينما بالـنسبة لـ\InlineCode{\#define}
فإنها تُطبّق على كلّ دوال الملف، و هذا قد يكون عمليّا جدّا في بعض الحالات.

هل من مثال واقعيّ لاستخدام
\InlineCode{\#define} ؟\\
هذا ما لن تتأخّر هن فعله. عندما تفتح نافذة في
\textenglish{C}، قد تحتاج إلى تعريف ثوابت المعالج القبلي لتحديد أبعاد النافذة :
\begin{Csource}
#define WINDOW_WIDTH 800
#define WINDOW_HEIGTH 600
\end{Csource}
الفائدة هي أنّه إن أردت تغيير حجم الواجهة (لأنّها تبدو لك صغيرة جدّا)، فيكفي أن تغيّر
\InlineCode{\#define}
و تعيد ترجمة الشفرة.

لاحظ أنّ
\InlineCode{\#define}
تكون عادة في ملفات
\InlineCode{.h}
مع نماذج الدوال (بامكانك أن ترى
\InlineCode{.h}
الخاصّة بالمكتبات مثل
\InlineCode{stdlib.h}،
ستجد الكثير من
\InlineCode{\#define}).\\
\InlineCode{\#define}
إذن هي "مسهّلات وصول"، يمكنك تعديل حجم نافذه عن طريق تعديل
\InlineCode{\#define}
بدل الذهاب للبحث في الدوال عن الموضع الذي تفتح فيه النافذة لتعديل الأبعاد. هذا ربح وقت للمبرمج.

كملخّص، ثوابت المعالج القبلي تسمح بـ"إعداد" برنامجك قبل ترجمته. إنّها أشبه بطريقة إعدادات صغيرة.

\subsection{الـ\texttt{define} من أجل حجم جدول}
نستخدم كثيرا
\InlineCode{\#define}
من أجل تعريف حجم الجداول. نكتب مثلا :
\begin{Csource}
#define MAX_SIZE 1000
int main(int argc, char *argv[])
{
	char string1[MAX_SIZE], string2[MAX_SIZE];
	// ...
\end{Csource}
\begin{question}
  و لكن... كنت أعتقد أنّه لا يمكننا وضع متغيّر أو ثابت بين القوسين المربعين أثناء تعريف جدول ؟
\end{question}
نعم هذا صحيح، لكنّ
\InlineCode{MAX\_SIZE}
ليس متغيّرا و لا ثابتا. في الواقع لقد قلت لك، المعالج القبلي يحوّل الملف قبل الترجمة إلى :
\begin{Csource}
int main(int argc, char *argv[])
{
	char string1[1000], string2[1000];
	// ...
\end{Csource}
و هذا شيء صحيح.

بتعريف
\InlineCode{MAX\_SIZE}
بهذه الطريقة، يمكنك استخدامها لإنشاء جداول ذات حجوم محدّدة. إذا صارت في المستقبل غير كافية، فليس عليك سوى تعديل سطر
\InlineCode{\#define}،
إعادة الترجمة، و جداول
\InlineCode{char}
تأخذ القيمة الجديدة الّتي حددتها.

\subsection{الحسابات في الـ\texttt{define}}
من الممكن القيام بحسابات صغيرة في الـ\InlineCode{\#define}.\\
مثلا، هذه الشفرة تنشئ ثابتا
\InlineCode{WINDOW\_WIDTH}،
و آخر
\InlineCode{WINDOW\_HEIGHT}،
ثمّ ثالثا
\InlineCode{PIXELS\_NUMBER}،
الذي يحوي عدد البيكسلز المعروضة داخل النافذة (الحساب بسيط : العرض $\times$ الطول).
\begin{Csource}
#define WINDOW_WIDTH 800
#define WINDOW_HEIGHT 600
#define PIXELS_NUMBER (WINDOW_WIDTH * WINDOW_HEIGHT)
\end{Csource}
قيمة
\InlineCode{PIXELS\_NUMBER}
يتمّ استبدالها قبل الترجمة بالشفرة التالية :
\InlineCode{(WINDOW\_WIDTH * WINDOW\_HEIGHT)}،
أي (800*600)، و تعطينا 480000.
ضع دائما حساباتك بين قوسين كما أفعل من باب الاحتياط لكي تعزل العمليّة.

يمكنك القيام بكل العمليات القاعدية التي تعرفها : جمع (+)، طرح (-)، ضرب (*)، قسمة (/)، ترديد (\%).

\subsection{الثوابت مسبقة التعريف}
بالإضافة إلى الثوابت التي أنت عرّفتها، فإنه توجد ثوابت معرّفة من قِبَل المعالج القبلي.

كل من هذه الثوابت تبدأ و تنتهي برمزين
\textenglish{underscore} \InlineCode{\_}
(تجده في لوحة المفاتيح تحت الرقم 8 أعلى اللوحة بالنسبة للتخطيط
\textenglish{AZERTY}).
\begin{itemize}
  \item \InlineCode{\_\_LINE\_\_} : يعطي رقم السطر الحالي من الشفرة.
  \item \InlineCode{\_\_FILE\_\_} : يعطي إسم الملف الحالي.
  \item \InlineCode{\_\_DATE\_\_} : يعطي تاريخ ترجمة الشفرة.
  \item \InlineCode{\_\_TIME\_\_} : تعطي وقت ترجمة الشفرة.
\end{itemize}
قد تكون هذه الثوابت مفيدة للتحكم في الأخطاء، مثال :
\begin{Csource}
printf("Error in the line n° %d of the file %s\n", __LINE__, __FILE__);
printf("This file has been compiled on %s at %s\n", __DATE__, __TIME__);
\end{Csource}
\begin{Console}
Error in the line n° 9 of the file main.c
This file has been compiled on 13 Jan 2006 at 19:21:10
\end{Console}
