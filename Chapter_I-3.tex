\chapter{برنامجك الأوّل}
لقد قمنا بحضير كلّ شيء إلى حد الآن ويمكننا أن نبدأ قليلا من البرمجة. مع نهاية هذا الدرس ستكون قد نجحت في انشاء أوّل برنامج لك.

لكي أصدقك القول، سيظهر البرنامج بالأبيض والأسود ولن يقوم بشيء سوى إلقاء التحيّة. يبدو عديم الفائدة، لكنّه برنامجك الأوّل وأؤكّد لك أنّك ستكون فخورا به.

\section{كونسول أو نافذة؟}
لقد تحدثنا سابقا عن فكرة برامج الكونسول وبرامج النوافذ في الدرس السابق. البيئة التطويرية تطلب منا تحديد أي نوع من البرامج نريد أن ننشئها. ولقد قلنا إننا سننشئ برامج من نوع كونسول.

يوجد نوعان من البرامج، لا أكثر:
\begin{itemize}
  \item برامج بنوافذ
  \item برامج تعمل في الكونسول
\end{itemize}

\subsection{البرامج الّتي تملك نوافذ}
هي البرامج التي نعرفها جميعا. هذا مثال على برنامج من نوع نافذة، مثل الرسام.
\Picture{Chapter_I-3_Paint}
أعتقد أنّك تحب إنشاء برامج كهذه، لكنّ هذا ليس في مقدورك حاليا. في الواقع، إنشاء برامج بنوافذ هو أمر ممكن بلغة \textenglish{C}، لكنّ بالنسبة لمبتدئ، هذا أمر معقّد جدّا. كبداية، يستحسن إنشاء برامج الكونسول.

\begin{question}
  لكن ماذا يعنى برنامج
\textenglish{Console}
؟
\end{question}

\subsection{البرامج الّتي تعمل في الكونسول}
برامج الكونسول هي أول ما ظهر من برامج. في ذلك الوقت، شاشات الحواسيب لم تكن سوى بالأبيض والأسود، ولم تكن فعّالة لكي تتمكّن من رسم النوافذ كما هو الحال مع حواسيبنا حاليّا.

مرّ الزمن بسرعة وزادت شعبية الويندوز نظراً لبساطته إلى أن نسي كثير من الناس ما هي الكونسول.

لديّ خبر جيّد لك!
\textbf{الكونسول لم تمت بعد}!
 في الواقع،
\textenglish{GNU/Linux}
 قد أعاد الكونسول إلى الحياة. هذه صورة لكونسول على
\textenglish{GNU/Linux}.

\Picture{Chapter_I-3_Console}

مرعب! صحيح؟ لكن على الأقل عرفت ما هي الكونسول، وهذه بعض الملاحظات:
\begin{itemize}
  \item اليوم، يمكننا عرض الألوان في الكونسول. ليس كلّ شيء بالأبيض والأسود كما تتخيّل.
  \item الكونسول هو الأسهل من ناحية البرمجة بالنسبة للمبتدئين.
  \item أداة عالية الإمكانيّات إذا عرفنا كيف نستخدمه.
\end{itemize}

كما قلت لك، إنشاء برامج كونسول أمر سهل جدّا وملائم للمبتدئين (وهذا عكس برامج النوافذ). ليكن في علمك أيضا أنّ الكونسول قد تطوّرت وبإمكانها عرض الألوان، ولا شيء يمنعك من إضافة صورة خلفيّة لها.

\begin{question}
  وفي الويندوز ألا توجد
\textenglish{console}
 ؟
\end{question}

بلى، لكنّها مخفيّة لو صح القول. يمكنك فتحها بالذهاب إلى إبدأ
\InlineCode{Start}
 ثمّ ملحقات
\InlineCode{Accessories}
 ثمّ موجه الأوامر
\InlineCode{Command prompt}
 أو بالذهاب إلى إبدأ ثمّ تشغيل
\InlineCode{Run}
 واكتب فيها
\InlineCode{cmd}
 واضغط على موافق.

\Picture{Chapter_I-3_Console-Windows}

إذا كنت تستخدم نظام ويندوز، فاعلم بأن أولى برامجك ستكون في نوافذ شبيهة بهذه. أنا لم أختر البداية هكذا لجعلك تشعر بالملل، بل لتعليمك الأساسيّات اللازمة لكي تتمكّن لاحقا من إنشاء النوافذ.

إذن فلتكن متيقّناً، بمجرّد أن تصل إلى المستوى اللازم لإنشاء النوافذ، سوف أعلّمك كيف تفعل ذلك.

\section{الحدّ الأدنى من الشفرة المصدرية}
من أجل أي برنامج، يجب كتابة قدر معيّن من الشفرة المصدرية. هذه الشفرة لا تقوم بشيء خاصّ لكنّها ضروريّة. هذه الشفرة التي سنكتشفها الآن ستكون أساس أغلب برامجك الّتي ستكتبها بلغة \textenglish{C}.

\subsection{أطلب من البيئة التطويرية الخاصة بك تزويدك بالحد الأدنى من الشفرة المصدرية}
لقد لاحظت أن طريقة إنشاء مشروع جديد تختلف من بيئة تطويرية إلى أخرى. إليك تذكيراً بسيطا: في برنامج الـ
\textenglish{Code::Blocks}
 (والذي سنستخدمه في هذا الكتاب)، عليك التوجه نحو
\InlineCode{File}
 ثمّ
\InlineCode{New}
 ثمّ
\InlineCode{Project}
 ثم تختار
\InlineCode{Console Application}
 وبعدها اللغة
\textenglish{C}
. سيولّد لك الحد الأدنى من الشفرة المصدرية
 \textenglish{C}
 التي تحتاجها. ها هي:

\begin{Csource}
#include <stdio.h>
#include <stdlib.h>

int main()
{
    printf("Hello world!\n");
    return 0;
}

\end{Csource}

\begin{information}
لاحظ أنّه يوجد سطر فارغ في نهاية الشفرة. يفترض أن ينتهي كل ملف مكتوب بلغة C هكذا. إن لم تفعل ذلك، فهذه ليست بمشكلة، لكن توقّع أن يعرض لك المترجم تحذيراً
(\textenglish{Warning}).
\end{information}

علماً أنّ السطر:
\begin{Csource}
  int main()
\end{Csource}

…بإمكانه أن يُكتب كالتالي:
\begin{Csource}
  int main(int argc, char *argv[])
\end{Csource}

كلتا العبارتين تحملان نفس المعنى لكن الثانية، الأكثر تعقيدا، هي الأكثر شيوعا، لذلك فإنّنا سنستخدمها في الدروس القادمة.\\
إستخدامنا للشكل الأوّل أو الثاني لا يغيّر شيئا بالنسبة لنا. لذلك لا داعي لإضاعة الوقت هنا، خصوصاً أنّك لا تملك المستوى اللازم لفهم ما تعنيه.

إذا كنت تستخدم بيئة تطويرية أخرى فقم بنسخ هذه الشفرة المصدرية وألصقها في الملف \InlineCode{main.c} ليكون لديكم نفس الشفرة.

أخيرا، قم بحفظ عملك في المشروع. أعلم أننا لم نقم بشيء حتّى الآن لكن من الجيّد التعوّد على الحفظ في كلّ مرّة.

\subsection{تحليل أسطر الشفرة المصدرية السابقة}
قد تبدو لك الشفرة المصدرية السابقة أنّها كاللغة الصينيّة، أنا أتخيّل ذلك! في الواقع هي تسمح بإنشاء برنامج كونسول يعرض نصّا على الشاشة. يجب تعلّم كيفيّة قراءة كلّ هذا.

فلنبدأ بأوّل سطرين:
\begin{Csource}
#include <stdio.h>
#include <stdlib.h>
\end{Csource}

هذان السطران يبدآن بعلامة
\InlineCode{\#}.
وهي أسطر خاصّة تُعرف باسم
\underline{توجيهات المعالج القبلي}
(\textenglish{Preprocessor directives})
. اسم معقّد، أليس كذلك؟ هذه الأسطر تتمّ قراءتها من طرف البرنامج المسمّى بالمعالج القبلي، وهو برنامج يتمّ تشغيله في بداية الترجمة.

ما رأيناه سابقا كان مخطّطا بسيطا لعمليّة الترجمة. لكنّ في الواقع، هناك الكثير من المراحل التي تحدث في هذه العمليّة. سنقوم بتفصيل هذا لاحقا. حاليّا عليك فقط تذكّر وضع هذين السطرين أعلى كلّ ملفّاتك.

\begin{question}
  حسنا لكن ماذا يعنيه هذان السطران؟ أريد أن أعرف!
\end{question}

كلمة
 \InlineCode{include}
 بالإنجليزيّة تعني "تضمين". هذان السطران يقومان بتضمين ملفّات في المشروع، أي إضافة هذه الملفّات من أجل عمليّة الترجمة. هناك سطران وبالتالي هناك ملفان يتمّ تضمينهما في المشروع وهما بالترتيب:
\InlineCode{stdio.h}
 و
\InlineCode{stdlib.h}.
هذان الملفّان موجودان بالفعل على حاسوبك وهما ملفّان مصدريّان جاهزان، سوف تعرف مستقبلا أنّنا نسميها
\underline{مكتبات}
(\textenglish{Libraries}).
 هذه الملفّات تحتوي الشفرة المصدرية اللازمة لعرض نصّ على الشاشة.

 بدون هذين الملفّين، كتابة نصّ على الشاشة سيكون أمرا مستحيلاً. فالحاسوب لا يعرف فعل أي شيء مبدئيا.

 باختصار، السطران الأول والثاني يقومان بتضمين المكتبات التي ستساعدنا في إظهار نصّ على الشاشة بكلّ سهولة.

 نمر للتالي، باقي الأسطر:
\begin{Csource}
int main()
{
    printf("Hello world!\n");
    return 0;
}
\end{Csource}

ما تراه هنا هو ما نسميه بـ
\underline{التابع}
أو
\underline{الدالّة}
(\textenglish{Function}).
 البرنامج في لغة
\textenglish{C}
 يتكوّن من مجموعة دوال. حاليّا برنامجنا لا يحوي سوى دالّة واحدة.

الدالّة تمكّننا من تجميع مجموعة من الأوامر. الغرض من تجميع الأوامر هو جعلها تقوم بوظيفة ما. مثلا يمكننا إنشاء دالّة باسم
 \InlineCode{open\_file}
 وجعلها تحتوي التعليمات التي تشرح للحاسوب كيفيّة فتح ملف.

 دون الدخول في تفاصيل إنشاء الدالّة (الوقت مبكّر، سوف نتحدّث عن الدوال في وقت لاحق) لنحلّل رغم ذلك أجزائه الكبيرة. السطر الأوّل يحتوي اسم الدالّة، إنّه الكلمة الثانية.\\
 أجل، اسم دالّتنا هو
\InlineCode{main}
والذي يعني
"الرئيسية"
. وتشغيل البرنامج دائما يبدؤ من الدالة
\InlineCode{main}.

للدالّة بداية ونهاية، وهي محدودة بالحاضنتين
\InlineCode{\{}
و
\InlineCode{\}}
. محتوى الدالّة موجود بين هاتين الحاضنتين. إن كنت قد تابعت جيداً فقد عرفت أنّ الدالّة مشكّلة من سطرين:

\begin{Csource}
printf("Hello world!\n");
return 0;
\end{Csource}

هاته الأسطر في الداخل نسميها
\underline{التعليمات}
(\textenglish{Instructions})
 (هذه إحدى المصطلحات الّتي يجب عليك حفظها). كلّ تعليمة تمثّل أمراً بالنسبة للحاسوب. فكلّ واحدة منها تطلب منه فعل شيء محدّد.

 كما قلت لك، بتجميع ذكيّ للتعليمات في الدالّة يمكننا إنشاء أجزاء برنامج جاهزة للاستخدام. باستخدام التعليمات المناسبة يمكننا إنشاء دالّة
 \InlineCode{open\_file}
 كما شرحت لك قبل قليل، و أيضا دالّة
\InlineCode{move\_character}
 في لعبة فيديو، على سبيل المثال.

 البرنامج في الواقع ما هو إلّا تتابع لتعليمات : إفعل هذا و إفعل ذاك. أنت تعطي أوامر للحاسوب و هو يقوم بتنفيذها.

 \begin{critical}
هامّ جدّا : لا بدّ أن تنتهي كلّ تعليمة بفاصلة منقوطة
"\InlineCode{;}"
. بهذا يمكن التفريق بين ما إذا كانت هذه تعليمة أم لا. إذا نسيت وضع فاصلة منقوطة نهاية تعليمة ما، فلن تتمّ ترجمة برنامجك.
 \end{critical}

 السطر الأول :
 \InlineCode{printf("Hello world!\\n");}
 يطلب إظهار الرسالة
 "\textenglish{Hello world!}"
  على الشاشة. عندما يصل برنامجك إلى هذا السطر، فسوف يقوم بعرض هذه الرسالة ثمّ المرور إلى التعليمة التالية.

  التعليمة التالية هي
\InlineCode{return 0;}
 و هي تخبرنا أنّ الدالّة
\InlineCode{main}
 قد انتهت و تطلب منه إعادة 0.

 \begin{question}
   لماذا يقوم برنامجي بإعادة العدد 0 ؟
 \end{question}

 في الواقع، كلّ برنامج عندما ينتهي يُرجع قيمة معينة. على سبيل المثال، ليقول أنّ كلّ شيء سار على ما يرام. عمليّا، 0 يعني كلّ شيء سار على ما يرام، و كلّ قيمة أخرى تدلّ على حدوث خطأ. في أغلب الأحيان هذه القيمة لا تُستخدم ، لكن يجب رغم ذلك استعمالها.\\
 كان يمكن أن يعمل برنامجك بدون
 \InlineCode{return 0}
، لكن يمكننا القول أن وضعها يعتبر أمراً أكثر نظافة و أكثر جدّية.

إلى هنا نكون قد فصّلنا قليلا في عمل هذه الشفرة المصدرية.

طبعا، نحن لم ندرس كلّ شيء بعمق، و قد تكون لديك بعض الأسئلة عالقة في ذهنك. كن على يقين بأنك ستجد لها أجوبة شيئا فشيئا مع تقدّمنا في الدروس. لا يمكنني أن أطلعك على كلّ شيء من البداية، لأنّ هناك كثيراً من الأشياء لاستيعابها.

إليك ما يلي: بما أنني في حال جيّدة، سأقوم بوضع مخطّط يضمّ المصطلحات الّتي تعلّمناها في هذا الدرس.
\Picture{Chapter_I-3_HelloWorld}

\subsection{لنجرّب برنامجنا}
كلّ ما سنقوم به الآن هو ترجمة المشروع ثمّ تشغيله (اضغط على
\InlineCode{Build \& Run}
 إذا كنت على
\textenglish{Code::Blocks}
). سيطلب منك حفظ مشروعك إذا لم تقم بذلك من قبل.

\begin{critical}
  إن لم تنجح الترجمة و ظهر لك خطأ مثل :\\
\InlineCode{"My-program - Release" uses an invalid compiler. Skipping...}\\\InlineCode{Nothing to be done...}
فهذا يعني أنّك نزلت نسخة
Code::Blocks
 دون
\InlineCode{mingw}
 (المترجم)، عد و حمّل النسخة التي تحتوي
\InlineCode{mingw}.
\end{critical}

بعد بُرهة، يظهر برنامجك كما في الصورة :
\Picture{Chapter_I-3_HelloWorld-run}

البرنامج يُظهر
"\textenglish{Hello world!}"
 (في السطر الأوّل).\\
الأسطر الّتي أسفله تمّ توليدها من طرف
\textenglish{Code::Blocks}
 وتدلّ على أنّ البرنامج قد تمّ تشغيله بنجاح كما أنها تعطي الوقت الذي استغرقه البرنامج في التشغيل.

 سيطلب منك الضغط على إحدى المفاتيح لإغلاق النافذة. أعلم أن الأمر لم يكن ممتعا جدّا. لكنه برنامجك الأوّل، وهذه لحظة ستتذكرها طيلة حياتك! ألا تعتقد ذلك؟

\section{كتابة رسالة على الشاشة}
من الآن سنقوم بإدخال التعديلات على الشفرة المصدرية السابقة. مهمّتك، إن قبلتها : عرض رسالة
"\textenglish{Good morning}"
 على الشاشة.

\begin{question}
  كيف يمكنني اختيار النص الّذي سيظهر على الشاشة؟
\end{question}

الأمر بسيط جدا، إذا بدأت من الشفرة التي رأيناها سابقاً، فسيكون عليك استبدال
"\textenglish{Hello world!}"
 بـ
"\textenglish{Good morning}"
 في السطر الذي يستدعي
\InlineCode{printf}.

كما قلت من قبل،
\InlineCode{printf}
 هي
\underline{تعليمة}
 وهي تعطي أمراً للحاسوب: "قم بعرض هذه الرسالة على الشاشة".\\
يجب أن تعرف أيضا أن
\InlineCode{printf}
 هي دالّة كُتِبَت من قبل من طرف مبرمجين قبلك.

\begin{question}
   أين توجد هذه الدالّة؟ أنا لا أرى سوى الدالّة \InlineCode{main} !
\end{question}

هل تذكر هذين السطرين؟
\begin{Csource}
#include <stdio.h>
#include <stdlib.h>
\end{Csource}
قلت لك من قبل أنهما يمكنان البرنامج من إضافة مكتبات. المكتبات في الحقيقة هي ملفّات تحوي أطنانا من الدوال جاهزة للإستخدام. هذه الملفات
(\InlineCode{stdio.h} و \InlineCode{stdlib.h})
 تحوي أغلب الدوال الأساسية التي قد نحتاجها في برنامج ما.
\InlineCode{stdio.h}
 بحد ذاته يحوي دوال تمكّن من عرض أشياء على الشاشة (مثل
 \InlineCode{printf}
) و أيضا الطلب من المستخدم إدخال شيء ما (هذه دوال سنتعرّف عليها لاحقا).
