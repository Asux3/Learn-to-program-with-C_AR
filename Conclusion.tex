\chapter*{خاتمة}

هل تريد
\textit{المزيد}
؟

لماذا لا تتعلّم لغة الـ\textenglish{C++}
؟

\url{http://www.siteduzero.com/tuto-3-5395-0-apprenez-a-programmer-en-c.html}

 هذا درس آخر كتبتُه حول هذه اللغة قريبة الـ\textenglish{C}.
 إذا كنت تعرف الـ\textenglish{C}،
فلن تكون ضائعاً بل ستفهم بسرعة فائقة الفصول الأولى !\\
فليكن في علمك أنني كتبت درساً قصيرا يسمّى "من الـ\textenglish{C}
إلى الـ\textenglish{C++}"
الذي يبيّن جزءً من الاختلاف بين الـ\textenglish{C}
و الـ\textenglish{C++}.

\url{http://www.siteduzero.com/tutoriel-3-430167-du-c-au-c.html}

بلغة الـ\textenglish{C++}،
يمكنك البدء في البرمجة غرضية التوجّه (أو البرمجة الكائنية) (\textenglish{OOP}).
قد يكون هذا المبدأ معقّدا قليلا في البداية، لكن ستجد بأن هذه الطريقة في البرمجة ناجعة جداً ! ستكتشف أيضاً معها المكتبة
\textenglish{Qt}
التي تسمح بإنجاز واجهات رسومية كاملة جدّا.

أشكر كثيرا
\href{http://www.siteduzero.com/membres-294-45753.html}{Taurre}
و
\href{http://www.siteduzero.com/membres-294-181268.html}{Pouet\_forever}
لمساعدتهم الكبيرة في القيام بالمراجعات الأخيرة لهذه الدروس.
