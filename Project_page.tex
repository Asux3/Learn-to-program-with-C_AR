\clearpage
\thispagestyle{empty}
\oldsection*{\LARGE\color{section}
تنزيل المشروع}

{\large
تمّت إنشاء هذا الكتاب بلغة التوصيف
\LaTeX
وترجمته بمترجم
\XeLaTeX.
يمكن الحصول على النص المصدريّ الخاصة به عن طريق استنساخ مستودع
\textenglish{GitHub}
الآتي:

\url{https://github.com/Hamza5/Learn-to-program-with-C_AR}

يوجد في هذه الصفحة أيضا رابط لتنزيل النسخة الرقميّة بصيغة
\textenglish{PDF}،
وشرح لطريقة الترجمة والاعتماديّات الواجب توفّرها، بالإضافة إلى النص المصدريّ الخاصة به.

إذا كنت من مستخدمي
\textenglish{GitHub}،
يمكنك التبليغ عن الأخطاء الّتي قد تجدها في الكتاب عن طريق فتح
\textenglish{issue}
في هذا المستودع وكتابة تفاصيل الخطأ (الفصل، القسم، الفقرة، رقم الصفحة والتصحيح الموافق إن أمكن)؛ أو عن طريق استنساخ/\textenglish{fork}
المستودع لإنشاء نسخة مطابقة كمستودع خاص بك، ثم إدخال التعديلات المرادة. بعد ذلك، يمكنك طلب دمج/\textenglish{pull request}
مستودعك بالمستودع الأصلي. في حالة ما كان التعديل جيّدا، سأقوم بدمجه في المستودع.
}

\oldsection*{\LARGE\color{section}
الترخيص
}
{\large
محتوى هذا الكتاب مرخّص تحت بنود رخصة
\textbf{المشاع الإبداعي، نسب المصنف - غير تجاري - الترخيص بالمثل، النسخة الثانية
(\textenglish{CC-BY-NC-SA 2.0})}،
تماما مثل ترخيص الدرس الأصلي المتوفّر في موقع
\textenglish{OpenClassrooms}.

\url{https://creativecommons.org/licenses/by-nc-sa/2.0/}

هذا يعني أنّه بإمكانك الاستفادة من هذا العمل، نسخه وإعادة توزيعه بأيّة وسيلة أو صيغة، وكذلك تعديله واستخدامه في أعمال أخرى. كلّ هذا بشرط أن تشير إلى العمل الأصلي، تعطي رابطا إلى هذه الرخصة، وتدلّ على التعديلات إن قمت بذلك. بالإضافة إلى ذلك، لا يمكنك استخدام عملك للأغراض التجاريّة من دون إذن صاحب العمل، كما يجب عليك ترخيص عملك بنفس الرخصة من دون فرض أيّة قيود إضافيّة على مستخدمي عملك.
